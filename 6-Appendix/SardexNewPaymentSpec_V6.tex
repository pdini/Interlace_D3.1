\chapter*{Appendix: Complete Functional Requirements\\ and Business Logic Model (2018)}
\label{appendix}

\vspace{-1cm}
\begin{center}
Egon B\"orger, Luca Carboni and Paolo Dini
\end{center}

We specify here the refinement of the abstract model defined in Deliverable D2.1 \cite{INTERLACE_D21} for the Sardex core payment operations Credit, Debit and B2C (in Euro or SRD).\footnote{We skip the account history and balance operations explained in Section 3.2 of D2.1, because they seem not to be in the focus anymore.} The refinement moves towards what is needed for an executable version of the model, though it still stays at the functional requirements level of abstraction. It is based upon the additional information obtained in the meantime on details of various system components, on specific data concerning groups, accounts and transactions, and on the resulting intended definition of permission features. This information is taken from Chapter \ref{ch:UpdatedBLS} of this report (D3.1).
%of D3.1.\footnote{We refer to the new version (end of June 2018) of this document. The differences of the model described here and its previous version of June 17, 2018 are due to a revised interpretation and definition of the TT function below. The necessity for this revision was found by inspection of the model of June 17.}

Section \ref{sect:signature} describes the refinement of the signature elements that were introduced in D2.1. Section \ref{sect:creditops} and Section \ref{sect:debitops} refine the model for the basic credit and debit operations considered in D2.1,  Section \ref{sect:userops} adds the new operations between companies and consumers, called B2C operations. Section \ref{sect:usrops} describes the users' input operations. To simplify the model inspection, in a sub-appendix (Section \ref{sect:appendixModel}) we put all rules of the model together in a nutshell (without the explanatory text which is to be found in the preceding sections where the rules are introduced).  Unexplained terminology and notation are used with the meanings explained in D2.1.

\section{Signature Elements}
\label{sect:signature}

In this section we describe the refined signature that is used for $Credit$, $Debit$ and B2C operations. The complex $transferTypeCheck$ function, defined in D2.1 in abstract terms to specify the $Credit$ and $Debit$ operations, is refined by splitting it into two independent checks (which may be thought of as executed in sequence, as will happen in the implementation).  The first check concerns the involved source and target groups and is again called TransferTypeCheck (see Section \ref{sect:usergroups}). The second check concerns the types of the source and destination accounts involved and is called AccountConnectivityCheck (see Section \ref{sect:accounts}).

This refinement is essentially a data refinement and concerns mainly 
\begin{itemize}
	\item user groups with their characteristic attributes (called \emph{group profile metadata}) and the constraints on the groups among which transfers are allowed by the system (called \emph{transfer type constraints}), and
	\item accounts with their charateristic attributes (called \emph{account metadata}) and the constraints on types of accounts between which an operation is allowed by the system (called \emph{account connectivity constraints}).
\end{itemize}

\subsection{User groups: profile metadata and transfer type constraints}
\label{sect:usergroups} 

{\bf Groups.} Out of the 29 types of users, which appear in the specification of the Sardex system as agents (actors) that interact with the system, following Chapter \ref{ch:UpdatedBLS} the refined model considers the following 9 pairwise disjoint dynamic sets, called \emph{groups}, which are characterized by the indicated specific attributes:

\begin{asm}
Company \mbox{  // set of actors which participate only in B2E and B2B operations}\\
Mngr 
  \mbox{  // (singleton set of) a distinguished element 
  	acting for the Sardex company}\\
Retail \mbox{  // set of actors which participate only in B2C operations}\\
Full \mbox{  // set of actors with both Company and Retail functionality}\\ 
\\
Employee \mbox{  // set of actors working for a member of $Company$ or $Full$}\\
Consumer \mbox{  // set of individuals which participate only in B2CEur operations} \\
Consumer\_Verified \mbox{  // set of registered consumers with additional B2CSrd operation} \\ 
\\
Welcome \mbox{  // set of users which have joined but are not yet cleared to start trading}\footnote{To have joined means having signed the contract to become a member of $Company \cup Retail \cup Full$.}\\
On\_Hold \mbox{  // set of actors whose privileges have been suspended}\footnote{Members of $Retail$, $Company$, $Full$, $Employee$, and $Consumer$\_$Verified$ can be suspended, not the $mngr$. For $Welcome$ and $Consumer$ the concept does not apply since they are not allowed to transact yet in any case.}
\end{asm}


{\bf Notational convention.} To simplify the exposition, we treat singleton sets, like $Mngr=\{mngr\}$, sometimes as set (here $Mngr$) and sometimes as element (here $mngr$), depending on the context, hoping that this slight abuse of language does not create any ambiguity. 
In general, we write $x$ for elements of $X$ where $X$ is one of the above 9 groups. To prepare the step towards a natural implementation of the model, we treat $Welcome$ and $On\_Hold$ members as potential members of some of the other groups which have however a $Welcome$ or a $On\_Hold$ flag set, respectively.

For further reference we define $Group$ as the set of the above nine groups:
\begin{asm}
Group= \+
       \{Company,Mngr,Retail, Full, Employee,\+
             Consumer,Consumer\_Verified, Welcome,On\_Hold\}
\end{asm}

Since by the disjointness constraint each user is assumed to belong to exactly one group, there is a function $group(user)$ which yields the group to which $user$ belongs.

{\bf Group Profile Metadata.} Each group comes with a set of attributes (called metadata) that are characteristic for its members. They include the following five data types which are used in the refined model as `profile metadata' (besides the usual `identity metadata' described in Chapter \ref{ch:UpdatedBLS}, which provide the information on the ID of a group member, its e-mail address, phone numbers, also its legal name, address, gps, fiscal ID (VAT), etc.).
\begin{itemize}
	\item $Company$, $Retail$ (and therefore also $Full$) and $Welcome$ members (but not the $Mngr$) come with a $capacity$, a location whose value represents the maximum yearly SRD volume the member committed to \emph{selling}, with payment in SRD, when it stipulated its contract with the Sardex company. The date of the stipulation of $capacity$ is recorded in a location $capacityDate$.
	\begin{itemize}
		\item For a given $account$ (typically in CC, see below Section \ref{sect:accounts}), a derived parametrized location introduced in Chapter \ref{ch:UpdatedBLS} as belonging to AccountMetadata, $availableCapacity(account)$, is defined as follows:\footnote{The defining equation for $availableCapacity$ holds only for non-$Welcome$ members, which -- differently from $Welcome$ members (see Table \ref{tab:InitialAccountSets}) -- have a defined $account$; it holds for $Welcome$ members only once they have been assigned an $account$.} 
		\[availableCapacity(account)=capacity-saleVolume(account)\]
		where the dynamic function $saleVolume(account)$ indicates the current total volume of \underline{sales} per year\footnote{This implies that $availableCapacity(account)$ is implicitly parametrized by the year.} using that $account$.
	\end{itemize}

\item $Company$ (and therefore also $Full$) members and $Mngr$ come with a $creditPercent$ location whose value represents the percentage of payments accepted by the member in circuit currency (SRD, etc.) for transactions whose value is above 1000 Euro.

\item $Company$ and $Retail$ (and therefore also $Full$) members come with a $euroFee$ location. 
	\begin{itemize}
	\item For $Company$ its value indicates an $InterTradeEuroFee$ for inter-circuit sales in the non-Euro currency of the circuit (SRD for Sardex, VTX for Venetex, etc).
	\item For $Retail$ its value indicates a $B2CEuroFee$ for B2C sales in EUR.
	\item For $Full$ its value is a two-element set $\{InterTradeEuroFee,B2CEuroFee\}$. 
	\end{itemize}

A $B2CEuroFee$ is a function which yields the percentage of the total value of the B2C sale, to be paid by the selling $retailer$ to the Sardex company. 

An $InterTradeEuroFee$ can be of two kinds. Either it is a function that simply yields 3\% of the intertrade amount, regardless of the networks involved and their members. Or it is a dynamic function $fee(amount, network1, network2)$ which yields a percentage of the trade $amount$ that may depend on the networks involved. In the current model, only the buyer has to pay the fee, though it is contemplated for a future extension that also the seller will have to pay a fee.

\item Both $Full$ and $Retail$ members come with two locations: $rewardRate$ and $acceptanceRate$. The $rewardRate$ value is a function which yields the percentage of reward the member offers in SRD currency to consumers engaged in a B2CEur purchase with the member; similarly, the $acceptanceRate$ defines the percent rate of the total value of a consumer purchase at which the member accepts SRD currency.

\end{itemize}

The $transferTypeCheck$ function defined in D2.1 uses a $Match$ predicate which expresses constraints on a) the groups of the account owners for the considered operation and on b) the type of the two accounts involved, in addition to c) constraints on the value of a set of meta-data, which in D2.1 were called `custom fields'. The constraints on the groups involved are refined here by a function of the following type:
\begin{asm}
TT: Operation \times Currency \times Group \rightarrow \{G \mid G \subseteq Group\}
\end{asm} 
\noindent where $Operation = \{Credit,Debit\}$ and $Currency =\{SRD,EUR\}$.
To simplify the exposition, we follow Figure \ref{fig:stack} and define $TT$ by a case distinction, considering the given pair of the first two arguments, say  $op \in \{Credit,Debit\}$ and $cur \in \{SRD,EUR\}$. Formally,
\[TT(op,cur,group)=TT^{op,cur}(group).\]
For each group, say $fromGroup$ of buyers, $TT^{op,cur}(fromGroup)$ defines the set $toGroups$ of groups
of sellers
%\footnote{The definition for the previous version of D3.1 was as follows: For each group, say $fromGroup$,  $TT^{op,cur}(fromGroup)$ defines the set $toGroups$ of groups with whose members a member of $fromGroup$ is allowed to try to perform the indicated $op$eration in the indicated $cur$rency.}
whose members are allowed to receive a transfer (to one of their accounts) from (an account of) a 
$fromGroup$ member via the operation $op$ in the currency $cur$ (under appropriate constraints we specify below on the amount of the transfer and some metadata of the account involved).\footnote{In D2.1 the names $fromMemberGroup$ and $toMemberGroup$ were used, see the $Match$ predicate definition there.} We 
now define the requirements described for these functions.
%in the end-of-June version of D3.1.

{\bf Transfer Type Constraints for $TT^{Credit,SRD}$.} A $Credit$ operation in $SRD$ currency can be started only by members of one of the following groups (called source group or $fromGroup$ of the operation):
\begin{itemize}
	\item $Company$, and therefore also $Full$ and in particular $Mngr$,
	\item $Employee$,
	\item $Consumer\_Verified$.  
\end{itemize}

For SRD Credit operations the following groups are allowed as target group (read: group of the member receiving the SRD credit, also called $toGroup$\footnote{See Figure \ref{fig:stack}.}):
\begin{itemize}
	\item every $Company \cup Full \cup Mngr$ member can trigger an SRD Credit operation to a member of $Company \cup Full \cup Mngr$ or of $Employee$, 

	
	\item every $Employee$  member can trigger an SRD Credit operation to a member of \newline $Company \cup Retail \cup Full$,
	
	\item every $Consumer\_Verified$ member can trigger an SRD Credit operation to a member of \newline $Retail \cup Full$.
\end{itemize}
This requirement is expressed in the refined ASM model by the following function definition:

\begin{asm}
TT^{Credit,SRD}(Company) \+
	 = TT^{Credit,SRD}(Full)\\
	 = TT^{Credit,SRD}(Mngr)\\
	 =\{Company,Full, Mngr,Employee\} \-
TT^{Credit,SRD}(Employee) =\{Retail,Company,Full\} \\
TT^{Credit,SRD}(Consumer\_Verified)=\{Retail,Full\}
\end{asm}	


Note that this definition, which will be used below for the transfer type check, allows no $Credit$ operation with target group $On\_Hold$. But note that  a $Retail$, $Company$ or $Full$ can have its $creditLimit$ set to 0 (by a broker operation we do not model here) which does not prevent that user from receiving credits for sales.

For ease of reference we say that a group member $MayAllowTransferForCreditOpns$ if it is an element of a group where $TT^{Credit,SRD}$ has a defined value:\footnote{Using the $group(mbr)$ function which denotes the unique group to which $mbr$ belongs, the definition can be equivalently expressed by:
  \begin{asm}
 MayAllowTransferForCreditOpns(mbr) \iff
    group(mbr) \in \{ Company, Full, Mngr, Employee, Consumer\_Verified \}
    \end{asm} }
 \begin{asm}
 MayAllowTransferForCreditOpns(mbr) \iff \\ \qquad\qquad\qquad\qquad\qquad\qquad\qquad
    mbr \in Company \cup Full \cup Mngr \cup Employee \cup Consumer\_Verified
    \end{asm}

{\bf Transfer Type Constraints for $TT^{Debit,SRD}$.} 


A $Debit$ operation in $SRD$ currency can be started only by members of one of the following groups (called again source group or $fromGroup$ of the operation):
\begin{itemize}
	\item $Retail$, $Company$, and therefore also $Full$ and in particular $Mngr$,
	\item $Employee$,
	\item $Consumer\_Verified$.  
\end{itemize}
For SRD Debit operations the following groups are allowed as target groups (again called $toGroup$\footnote{See Figure \ref{fig:stack}.}):
\begin{itemize}
	\item Every $Company \cup Full \cup Mngr$ can initiate an SRD Debit operation that draws on a member of $Company \cup Full$ (and the latter could initiate a corresponding credit transfer to the former, for the same effect).
	\item The $mngr$ can initiate an SRD Debit operation that draws on any member of $Retail$.
	\item Every member of $Retail \cup Company \cup Full$ can initiate an SRD Debit operation that draws on any $Employee$.
	\item Every member of $Retail \cup Full$ can initiate an SRD Debit operation that draws on any member of $Consumer\_Verified$.
	\item Every member of $Company \cup Full$ can initiate an SRD Debit operation that draws on $mngr$.
\end{itemize}
This requirement is expressed in the refined ASM model by the following function definition:

\begin{asm}
	TT^{Debit,SRD}(Company) \+
	= TT^{Debit,SRD}(Full)\\
	=\{Company,Full, Mngr\} \-
	TT^{Debit,SRD}(Retail)=Mngr \\
	TT^{Debit,SRD}(Employee) =\{Retail,Company,Full\} \\
	TT^{Debit,SRD}(Consumer\_Verified)=\{Retail,Full\}\\
	TT^{Debit,SRD}(Mngr)=\{Company,Full\}
\end{asm}	


Correspondingly we define the domain predicate of $TT^{Debit,SRD}$:
\begin{asm}
\label{debitdomaindef}
MayAllowTransferForDebitOpns(mbr) \iff\+
 mbr \in Retail  \cup Company \cup Full \cup Employee \cup Consumer\_Verified \cup Mngr 
\end{asm}

{\bf Transfer Type Constraints for operations in Euro.} In the Sardex system, no $Credit$ operations are allowed in EUR currency, so that no function $TT^{Credit,EUR}$ is needed.\footnote{Table \ref{tab:TTs} includes the function but shows that it is completely $\UNDEF$ined.} A EUR $Debit$ operation can be triggered only by a $Consumer$ or a $Consumer\_Verified$ member and must have as target group $Retail$ or $Full$. This requirement is expressed in the refined ASM model by the following definition:
\begin{asm}
T^{Debit,EUR}(Consumer)=T^{Debit,EUR}(Consumer\_Verified)\+
  = \{Retail,Full\}
\end{asm}

In the present formulation of the model we do not use the function $T^{Debit,EUR}$ because, to simplify the exposition in Section \ref{sect:userops}, we describe the B2C operations there directly, not as instance of the $\ASM{DebitTransferReq}$ rules.

\vspace{12pt}

{\bf Remark on the refinement.} With the above definition of functions $TT^{op,cur}$, the transfer type part of the $Match$ predicate in D2.1 can be expressed as follows. By definition, the $cur$rency parameter of the $amount$ in question can be computed from the $account$ by the function $cur(account)$ (for accounts see Section \ref{sect:accounts}).
Furthermore, given that each member belongs to exactly one group, from the account parameters $from/to$ we also obtain $group(owner(from/to))$ in the following refinement of the group type constraints, namely $owner(from) \in fromGroup(tt) $ and $owner(to) \in toGroup(tt)$: 


\begin{asm}
\LET transfer = (op,channel,from,to,amount,metaData) \+
TTMatch(transfer) \IFF \+
\FORSOME g   \in	TT(op,cur(from), group(owner(from)) ) ~~~~  
    owner(to) \in g
\end{asm}

Note that the transfer type check is independent of the $channel$ used for a transfer.\footnote{The model in D2.1 was based on the assumption that channels could play a role for the transfer type check, though the constraints had not been specified. Therefore an abstract channel condition appeared there in the $Match$ predicate in D2.1, which is not present anymore here.}



\subsection{Account types, account metadata and account connectivity constraints.}
\label{sect:accounts}

{\bf Account Types.} Each $user$ may have a set $Account(user)$ of $acc$ounts each of which the $user$ is the  $owner(acc)$ of (also denoted $MemberId(acc)$), but at most one account per account type. Each $acc$ount is in one currency $cur(acc) \in Currency=\{SRD,EUR\}$ and of an $accountType(acc)$, defined for the group to which the $user$ belongs, among the following seven ones that are considered in the refined model. There are two SRD account types:
\begin{itemize}
	\item CC: the set of standard Sardex credit accounts, in SRD
	\item Domu:  the set of Sardex credit accounts for larger operations, in SRD
\end{itemize}

There are three types of EUR accounts, which are considered to be of statistical character:
\begin{itemize}
	\item  Income: the set of (statistical) accounts owned by $Retail$ or $Full$ users, which use such an account to record their collection of B2C payments in Euro.
	\item Prepaid: the set of (statistical) accounts from which the Euro transaction fee for a B2C operation is drawn by $SysAdmin$ on behalf of the $mngr$.\footnote{Differently from $Income$ accounts, any $Prepaid$ account is updated not by the $retail$er, for which it keeps track of its fee prepayments, but by the $mngr$. See Figure \ref{fig:B2C1} (B2C Use Case 1/2) and Section \ref{sect:Prepaid} for details.} Prepaid is owned by $Retail$ or $Full$ members but controlled by $SysAdmin$.	
	\item Bisoo: the set of (statistical) accounts in Euro which are used by $consumer$s to pay into an Income account. Bisoo is owned by $Consumer$ or $Consumer$\_$Verified$ members but controlled by $SysAdmin$.
\end{itemize}

There are two special $mngr$ account types:
\begin{itemize}
	\item Mirror: type of account in circuit currency (SRD, VTX, etc.) used for inter-circuit purchases. There is one mirror account per circuit so that, formally, each Mirror account is implicitly parametrized by a circuit.	
	\item Topup: a (statistical) account in Euro, which is $mngr$-owned and $SysAdmin$-controlled\footnote{See Figure \ref{fig:B2C1} (B2C Use Case 1/2).}  and used to recharge a $retail$er's Prepaid account upon receipt of a (real) payment in Euro (through normal banking channels).
\end{itemize}

We treat each account type as the set of accounts of that type and say that $accountType(acc)$ is $X$ if $acc \in X$, where $X$ is one of the seven account types above.\footnote{In D2.1 the names $creditAccount$, $domuAccount$, $feeAccount$ were used instead of $CC$, $Domu$, $Prepaid$.} We also write $x$ for elements of $X$, where $X$ is one of the above 7 account types. For the singleton set $Topup=\{topup\}$ we use the same notational convention as explained above for $Mngr=\{mngr\}$. For the set of the above 7 account types we write:
\[AccountType=\{CC,Domu,Income,Prepaid,Bisoo,Mirror,Topup\} \]

Initially, accounts are assigned to users satisfying the following constraints on $Account(x)$, depending on the user group $X$ the user $x$ belongs to:\footnote{The accounts of type $Prepaid$, $Bisoo$ and $Topup$ are not controlled by the corresponding member for which they serve but by the $SysAdmin$istrator, see below.
}

\begin{asm}
Account(company) = \{ cc(company), domu(company), prepaid(company) \} \\
\textcolor{red}{Account(company) = \{ cc(company), domu(company)\} } \\
Account(retail) = \{ cc(retail), prepaid(retail) , income(retail) \} \\
Account(full) = \{cc(full),domu(full), prepaid(full), income(full) \} \\ 
\\
 Account(mngr) = \{cc(mngr), topup \} \cup 
    \{mirror(circuit) \mid circuit \in Circuit\} \\
\\
Account(employee) = \{cc(employee)\} \\
Account(consumer) = Account(consumer\_verified)              
              = \{cc(consumer)\}\footnote{Contrary to Table \ref{tab:InitialAccountSets}, in this model no $Bisoo$ account is assigned since for simplicity we treat $consumer$-provided Euro payments directly as input, using a monitored $Received$ predicate. Then no $Bisoo$ account is needed. See Section \ref{sect:eurob2c}.}
\end{asm}

Remark. $Welcome$ members initially have no account yet (formally meaning  $Account(welcome)=\emptyset$). The initial accounts of a 
$user \in On\_Hold$ are those accounts the $user$ has at the moment it is placed into $On\_Hold$, but by being placed there the $user$ becomes unable to use these accounts.

{\bf Account Metadata.} Account metadata are used to formulate the constraints for allowed transfer $amount$s, which involve various account attributes like the $balance$, the $creditLimit$, the $upperLimit$, the $capacity$, etc. 

Every $acc$ount has the locations $owner(acc)$ and $curr(acc)$ (which is also called $unit(acc)$) introduced above. For $Bisoo$ and $Topup$ accounts these are the only locations needed in the refined model, so for them there are no other metadata. 

In addition, for each $acc$ount in $CC \cup Domu \cup Mirror$ there are the following locations, classified as metadata:

\begin{itemize}
	\item $balance$ (a Real number, modelled as 2-digit decimal)
	
	\item $creditLimit$ (a non-negative number)\footnote{Table \ref{tab:AccountMetaData1} includes also a $creditLimitDate$ location, indicating the date at which the credit limit was set. We skip such locations because no operation has been specified which uses them, so that there is no rule in the model which involves them.}
	
	\item $availableBalance$, a derived location required to be non-negative and defined by:
	\[availableBalance=balance+creditLimit.\]
\end{itemize}

In addition, $CC$ and $Mirror$ (but not $Domu$) accounts have the following  location:
\begin{itemize}
\item $upperLimit$ denoting the upper balance limit, a positive number.
\end{itemize}
In connection with $balance$ and $saleVolume$, there are three predicates which trigger an alert when the monitored item reaches its (low or high) bound:

\textcolor{red}{Question pending on the type of these predicates (Boolean? Integer? ... )}
\begin{asm}
lowBalanceAlert \IFF creditLimit+balance<lowBalanceAlert \+
   \mbox{ // small amount of money left in the account to buy something with}\-
highBalanceAlert \IFF upperLimit - balance<highBalanceAlert \+
        \mbox{ // small amount of space (measured in money unit) left for further sales}\-
highVolumeAlert \IFF capacity-saleVolume<highVolumeAlert \+
           \mbox{ // almost reached yearly committed sale volume}
\end{asm}

$Income$ and $Prepaid$ accounts have, besides  $owner(acc)$ and $curr(acc)$, also the $balance$ location. Note that a $Prepaid$ account has no $creditLimit$; otherwise stated, its credit limit is always 0.


{\bf Account Connectivity Constraints for user-initiated operations.} The account connectivity constraints serve to describe the account type conditions which are part 
of the $Match$ predicate defined in D2.1. To keep the model as simple as possible, 
we describe the intended effect of the new B2C operations in Section \ref{sect:eurob2c} separately, so that we can limit ourselves here to formulate the account connectivity refinement only for user-initiated transactions.

The requirements on the types of the accounts that may be involved in an operation state which account types are allowed as $toAcc$ for an operation of a given $fromAcc$. They can be formalized using an account type check $AccT$ analogous to the transfer type check function $TT$ above.

The essential requirement for user-initiated Credit or Debit operations is two-fold (see Figure \ref{fig:User_Acct_Connectivity}):

\begin{itemize}
	\item A $Credit$ operation in SRD that starts at a $fromAcc$ in $CC \cup Domu \cup Mirror$ has as allowed $toAcc$ a $CC$ account; in the case of $fromAcc \in CC$, also a $Domu$ or $Mirror$ account is permitted as $toAcc$.
	\item A $Debit$ operation in SRD can only start from a $fromAcc \in CC$ and must have a $toAcc \in CC$.
\end{itemize}

One can formalize the above requirements by defining the function 
\begin{asm}
AccT\colon Operation \times Currency \times AccountType \rightarrow \{Acct \mid Acct \subseteq AccountType\}
\end{asm} 

\noindent again by a case distinction $AccT(op,cur,accountType)=AccT^{op,cur}(accountType)$, where: 

\begin{asm}
AccT^{Credit,SRD}(CC) =\{ CC , Domu ,Mirror \}   \\
AccT^{Credit,SRD}(Domu) =AccT^{Credit,SRD}(Mirror)=\{CC\} \\
AccT^{Debit,SRD}(CC) =\{CC\}.
\end{asm}

 As above, for ease of reference we say that an account $MayStartCredit/DebitOpns$ if it is an element of an account type where $AccT^{Credit/Debit,SRD}$ has a defined value: 
\begin{asm}
MayStartCreditOpns(acct) \IFF acct\in CC \cup Domu \cup Mirror\footnote{For the reason explained in the next remark we exlude $Income$ accounts from this definition.} \\
MayStartDebitOpns(acct) \IFF acct\in CC.\footnote{Remember that we model here only Debit operations in SRD, not in EUR.}
\end{asm}


{\bf Remark on Bisoo.}
In Section \ref{subsec:perm-acc-con} also $Debit$ operations in EUR from an $Income$ to a $Bisoo$ account are allowed. Formally this means that in the definition above one includes the clause
\[AccT^{Debit,EUR}(Income) =\{Bisoo\}.\]
Since in rule $\ASM{EurB2C}$ below (Section \ref{sect:eurob2c}) we provide a direct formalization of the B2C operation, we can do without the $Bisoo$ account and without any $Debit$ operation in EUR from an $Income$ to a $Bisoo$ account.

{\bf Remark on system-initiated Credit/Debit operations.} In Figure \ref{fig:SysAdmin_Acct_Connectivity} the $AccT$ function is defined also for system-initiated operations. $AccT$ with first argument $Credit$ is defined by:
\[AccT^{Credit,SRD}(CC)=\{CC\}  \qquad  \AND  \qquad  AccT^{Credit,EUR}(Topup)=\{Prepaid\} \]

$AccT$ with first argument $Debit$ is defined by:
\[AccT^{Debit,EUR}(Topup)=\{Prepaid\}\]
\[ \textcolor{red}{AccT^{Debit,EUR}(Prepaid)=\{Topup\}} \]

{\bf Remark on the refinement.} With the above definitions one can redefine the account connectivity constraint part of the $Match$ predicate of D2.1, namely $sourceType(tt)=accountType(from)$ and $destType(tt)=accountType(to)$, as follows:
\begin{asm}
\LET transfer = (op,channel,from, to,amount,metaData) \+
AccTMatch(transfer) \IFF ~~   accountType(to) \in AccT(op,cur(from),accountType(from))
\end{asm}

\section{Credit Operation} 
\label{sect:creditops}

Here we define the (refined version from D2.1 of) user-initiated Credit operations in SRD. Similarly to user-initiated Debit operations in SRD, they use either a $Service$ channel (website or mobile phone) or a $PointOfSale$ ($POS$), the set of standard terminals used by retailers for EUR transactions or to route SRD transactions via an API. This is a mere renaming w.r.t.\ the terminology used at the time of writing of D2.1:
\begin{asm}
Channel=Service \cup POS \\
Service=\{website,mobilePhone\}
\end{asm}

In the refined model there is also a system-initiated Credit operation in Euro which involves $Topup$ and $Prepaid$ accounts. Since it does not involve the various transfer type checks, it can be modelled in a simpler way (see Section \ref{sect:userops}).

As in D2.1, a $\ASM{CreditTransferReq}$ initiated by a member $mbr$ (which is permitted only in SRD currency) splits into a double exchange of messages between the member and the Sardex system, called preview and perform step.

\begin{asm}
\ASM{CreditTransferReq}((channel,from, to, amount),mbr)=\+
   \ASM{CreditPreviewReq}((channel,from,to,amount),mbr)  \\
   \ASM{CreditPerformReq}((channel,from,to,amount),mbr) 
\end{asm}


A user-initiated Credit operation transfers an $amount$ of SRD via a specific $channel$ from one account $from$ to another $to$ under a certain number of conditions:

\begin{itemize}
	\item if the account owners belong to groups of permitted types, as specified by the transfer type check function $TT^{Credit,SRD}$, 
	\item if the involved accounts $from,to$ are of permitted types, as specified by  the account connectivity check function $AccT^{Credit,SRD}$, 
	\item if the $amount$ satisfies the constraints on the various limits, as specified by the $AccountLimitCheck$.
\end{itemize}


\subsection{$\ASM{CreditPreviewReq}$ program}
\label{sect:creditpreview}

The various constraints split again in a series of to-be-checked  more detailed conditions. To reflect the transactional nature of the $\ASM{CreditTransferReq}$ steps, we describe the execution of each of its two components as one atomic step. Nevertheless, to simplify the verification of the correctness and completeness of the rules, we formulate the entire check as successive If-Then-Else checks of all its single conditions. As a byproduct we obtain a detailed analysis of the possibilities for error handling procedures one may wish to implement. Thus we define  $\ASM{CreditPreviewReq}$ and $\ASM{CreditPerfomReq}$ using an instance of the $\ASM{IfThenElseCascade}$ pattern (see the Appendix in Section \ref{sect:appendix}). 

Furthermore, since it is required that the credit type checks in $\ASM{CreditPreviewReq}$ are repeated for $\ASM{CreditPerformReq}$, we make the nested credit type check pattern explicit as a machine $\ASM{CreditTypeCheck}$ which contains as parameter a final $\ASM{Completion}$ step. The $\ASM{Completion}$ component parameter can then be instantiated specifically for the two rules: for the preview step it comes up to inform the user that the $\ASM{CreditPerformReq}$ can be triggered, whereas for the perform step it specifies the required $\ASM{AccountLimitCheck}$. 


\begin{asm}
\ASM{CreditTypeCheck}(transfer,mbr,\ASM{Completion})  =\+
  \IF mbr =owner(from) \AND MayAllowTransferForCreditOpns(mbr) \+
   \THEN ~ \IF ~ \FORSOME g \in TT^{Credit,SRD}(group(mbr)) 
      ~~ owner(to) \in g   \mbox{  // check transfer type}\+
        \THEN  ~ \IF MayStartCreditOpns(from) 
        \mbox{  // check account connectivity}\+
             \THEN ~ \IF ~ accountType(to) \in 
                  AccT^{Credit,SRD}(accountType(from))\+
                  \THEN ~ \ASM{Completion}(transfer) \\
                  \ELSE ~ \ASM{Send}(ErrMsg(CreditTargetAccountViolation(transfer)),\TO mbr) \-
             \ELSE ~ \ASM{Send}(ErrMsg(CreditSourceAccountViolation(transfer)),\TO mbr)\-
       \ELSE ~ \ASM{Send}(ErrMsg(CreditTargetGroupViolation(transfer)),\TO mbr) \- 
  \ELSE ~  \ASM{Send}(NotAccountOwnerOrCreditSourceGroupViolation(transfer),\TO mbr)\dec\dec\-
\WHERE \+
   transfer = (credit, channel, from, to, amount)   
\end{asm}



Now we can define $\ASM{CreditPreviewReq}$ as a $\ASM{CreditTypeCheck}$ instance where the $\ASM{Completion}$ parameter is instantiated to what is needed here, namely to  $\ASM{PermitPerformReq}$ by $\ASM{Send}$ing a message that the $CreditPerformReq$ can be submitted successfully.

\begin{asm}
\ASM{CreditPreviewReq}((channel,from,to,amount),mbr)  =\+
  \LET transfer=(credit,channel,from,to,amount)\\
  \IF Received(CreditPreviewReq(transfer),\FROM mbr)\footnote{Given the account $owner$ function, one could omit here the $mbr$ parameter and write instead $owner(from)$.} \THEN \+   
      \ASM{CreditTypeCheck}(transfer,mbr,\ASM{PermitPerformReq})\\
      \ASM{Consume}(CreditPreviewReq(transfer)) \-
\WHERE \+
   \ASM{PermitPerformReq}(transfer)=\+
      \ASM{Send}
      (YouMayTriggerPerformReq(transfer),\TO owner(from))
\end{asm}

\subsection{ $\ASM{CreditPerformReq}$ program}
\label{sect:creditperform}


Similarly, the $\ASM{CreditPerformReq}$ rule uses an instance of  $\ASM{CreditTypeCheck}$ where the parameter $\ASM{Completion}$  is instantiated to a rule
$\ASM{CreditAccountLimitsCheck}$. In other words, to execute $\ASM{CreditPerformReq}$, first the $\ASM{CreditTypeCheck}$ is executed once more, but if it succeeds, to complete the operation, instead of $\ASM{Send} (YouMayTriggerPerformReq(transfer),\TO owner(from))$, a rule $\ASM{CreditAccountLimitsCheck}$ is called to $\ASM{TryToCompleteCreditOpn}$. That rule specifies the check of the various constraints on the $amount$ of the Credit operation, i.e.\ it refines the $balancecheck$ function of D2.1. Like $\ASM{CreditTypeCheck}$, it is an $\ASM{IfThenElseCascade}$ pattern instance and also comes with a $\ASM{Completion}$ parameter. For $\ASM{CreditPerformReq}$ this parameter is instantiated by a $\ASM{CompleteTransaction}$ component.

\begin{asm}
\ASM{CreditPerformReq}((channel,from,to,amount),mbr)  =\+
\LET transfer=(credit,channel,from,to,amount)\\
\IF Received(CreditPerformReq(transfer),\FROM mbr) \THEN \+  
  \ASM{CreditTypeCheck}(transfer,mbr,\ASM{TryToCompleteCreditOpn})\\
   \ASM{Consume}(CreditPerformReq(transfer))\-
\WHERE \+
  \ASM{TryToCompleteCreditOpn}(transfer)=\+
     \ASM{CreditAccountLimitsCheck}(transfer,\ASM{CompleteTransaction}(transfer))
\end{asm}

$\ASM{CreditAccountLimitsCheck}$ checks the following data for the given $amount$:
\begin{itemize}
	\item the $availableBalance$ of the source account $from$ (as already formulated in D2.1), where $from$ must be (and by the account connectivity check is known to be) a member of $CC \cup Domu \cup Mirror$,
	
	\item the $upperLimit$ of the target account $to$ (as already formulated in D2.1), where by the account connectivity check $to$ is known to be a member of $CC \cup  Mirror \cup Domu$, but by the account metadata definition cannot be an element of $Domu$ (accounts without $upperLimit$ location), 
	 
	 \item the $availableCapacity$ of the target account, where by the account metadata definition the target account (which by the account connectivity check is known to be an element of $CC \cup  Mirror \cup Domu$) must not be a $Domu$ account (because those accounts have no $availableCapacity$ location).
 \end{itemize} 

{\bf Remark on $creditPercent$.} The main use of the $creditPercent$ location is for statistical purposes, namely to compare the average EUR volume moved by the SRD volume in a given year. However, this feature is not currently implemented and will not be implemented in the CoreASIM model either.

%%%%%%% Note: this text is skipped for now as not immediately relevant to this model
%\textcolor{blue}{Currently the system does not check whether the amount of credits being moved, if the total amount in question is more than 1000, complies with the $creditPercent$ of the Seller, nor is the SRD part of the payment related to the payment of the remaining part of the total amount.} Thus, the current definition of $\ASM{AccountLimitsCheck}$ does not reflect the $creditPercent$ requirement mentioned in D3.1. \textcolor{red}{So what action should be taken? Should the rule stay or should it be changed to its current implementation?}

\begin{asm}
\ASM{CreditAccountLimitsCheck}(transfer,\ASM{StepCompletion})=\+
   \IF CanBeSpentBy(from,amount) \+
      \THEN ~ \IF CanBeCashedBy(to,amount) \+
         \THEN ~ \IF HasSellCapacityFor(amount,to) \+
            \THEN ~ \ASM{StepCompletion} \\
            \ELSE 
             ~ \ASM{Send}(ErrMsg(CapacityViolation(transfer)),\TO owner(from))\-
         \ELSE 
         ~ \ASM{Send}(ErrMsg(UpperLimitViolation(transfer)),\TO owner(from))\-
      \ELSE 
      ~ \ASM{Send}(ErrMsg(AvailBalanceViolation(transfer)),\TO owner(from))\dec\-
      \WHERE \+
transfer=(credit,channel,from,to,amount)\\
CanBeSpentBy(from,amount) \IFF availableBalance(from) \geq amount \\
CanBeCashedBy(to,amount) \IFF  
    to \not \in Domu \AND balance(to)+amount \leq upperLimit(to) \\
 HasSellCapacityFor(amount,to) \IFF  to \not \in Domu \AND   amount \leq availableCapacity(to)\footnote{Here we treat $availableCapacity$ as belonging to AccountMetaData, as stated in D3.1.}
\end{asm}


The $\ASM{CompleteTransaction}$ component still remains rather abstract, as in D2.1, until we obtain more information on the $Ledger$ and the used  $transaction$ function (which records the information on the $transfer$ that is appended to the $Ledger$, including a time stamp which we describe by a 0-ary system function $now$). However, by the knowledge of the transfer and account type functions $TT, accountType$ we can refine what in D2.1 was called the transfer type check result $ttResult$, namely the triple consisting of the group the $owner(to)$ belongs to and of the $accountType$ of the source and target accounts.

\begin{asm}
\ASM{CompleteTransaction}(transfer)=\+  
\LET (credit,channel,from,to,amount)=transfer \\ 
   \ASM{Append}(transaction(transfer,ttResult,now),Ledger)\\
   \ASM{Send}(Confirmed(transfer),\TO owner(from))\\
   saleVolume(to):=saleVolume(to)+amount\footnote{We treat the dynamic function $saleVolume$ as belonging to AccountMetaData, apparently in accordance with D3.1.} \-
\WHERE \+
 ttResult= (group(owner(to)),accountType(from),accountType(to) )
\end{asm}



{\bf Historical remark.} At the time of writing the model in D2.1, the understanding was that the (at the time otherwise not further specified) $custFlds$ parameter, which now would be renamed to $metadata$, plays a role for the $\ASM{CreditPreviewReq}$ rule. We now know that for the behavior of this rule only the group and account type properties are relevant so that the parameter can be skipped. Similarly for $\ASM{CreditPerformReq}$.




\section{Debit Operation} 
\label{sect:debitops}

As explained in D2.1, user-initiated Debit operations in SRD are executed in 3 phases: in addition to the $\ASM{DebitPreviewReq}$ and the $\ASM{DebitPerformReq}$ steps, where the system and the $creditor$ interact with each other similarly to the interaction for Credit operations, there is an interaction between the system and the $debitor$ where the system asks for an acknowledgement from the $debitor$ before performing the $\ASM{DebitAckReqAnswCompletion}$ (or in case of failure a $\ASM{DebitLateAckReqAnswCompletion}$) step. Therefore we have:

\begin{asm}
\ASM{DebitTransferReq}=\+
   \ASM{DebitPreviewReq} \\
   \ASM{DebitPerformReq} \\
   \ASM{DebitAckReqAnswCompletion}\\
    \ASM{DebitLateAckReqAnswCompletion}
\end{asm}

{\bf Remark on Debit operations in Euro.} In the refined model there are also two Debit operations in Euro. One is system-initiated and involves $Topup$ and $Prepaid$ accounts, the other one is user-initiated  and involves $Bisoo$ and $Income$ accounts. Since these two operations do not need the various transfer type checks, they can be modelled in a simpler way than by treating them as instances of $\ASM{DebitPreviewReq}$ (see Section \ref{sect:userops}).

\subsection{$\ASM{DebitPreviewReq}$ program (for SRD)}
\label{sect:debitpreview}

In this section we consider Debit operations in SRD. We treat the Debit operations in EUR separately in Section \ref{sect:userops}.

As for Credit operations, the $\ASM{DebitPreviewReq}$ step 
essentially performs a $\ASM{DebitTypeCheck}$ on the 
$creditor$ and $debitor$ groups and on the permitted type of 
the involved accounts. But the arguments of $TT$ are interchanged: for a Debit operation $TT$ is applied to  $group(debitor)$ and yields the allowed creditor groups, to one of which the $creditor$ must belong. By Fig. 2.3 of D3.1, the account connectivity check for SRD-Debit operations checks whether the accounts $from,to$ to-be-used for the intended Debit operation are both of type CC. In case the outcome of the two checks is positive, $\ASM{DebitPreviewReq}$ calls as completion a $\ASM{PermitPerformReq}$ component which enables the $creditor$ to proceed to the $\ASM{DebitPerformReq}$ phase. 


\begin{asm}
\ASM{DebitPreviewReq}((channel,from,to,amount),creditor)  =\+
  \LET transfer=(debit,channel,from,to,amount)\\
  \IF Received(DebitPreviewReq(transfer),\FROM creditor) \THEN \+   
      \ASM{DebitTypeCheck}
          (transfer,creditor,\ASM{PermitPerformReq})\\
      \ASM{Consume}(DebitPreviewReq(transfer)) \-
\WHERE \+
\ASM{PermitPerformReq}(transfer)=\+
\ASM{Send}(YouMayTriggerPerformReq(transfer)
\footnote{This message,  by its parameter $debit$, differs 
	from the message with same name $YouMayTriggerPerformReq$ used in the rule $\ASM{CreditPreviewReq}$ where the corresponding parameter is $credit$.}
,\TO  owner(to)\footnote{$creditor$, the sender of the 
	request, can be retrieved from $transfer$ by the account $owner$ function via $creditor=owner(to)$, which is the seller who will receive the transfer of SRD from the buyer $owner(from)$.}) 
\end{asm}

The $\ASM{DebitTypeCheck}$ machine follows the same pattern as $\ASM{CreditTypeCheck}$, notably with interchanged arguments for $TT^{Debit,SRD}$. 


\begin{asm}
\ASM{DebitTypeCheck}(transfer,creditor,\ASM{Completion})  =\+
\LET (debit,channel,from,to,amount) = transfer \\
\LET debitor=owner(from)\\
\IF creditor=owner(to) \AND MayAllowTransferForDebitOpns(debitor) 
      \+
  \THEN~ \IF~\FORSOME g \in TT^{Debit,SRD}(group(debitor))
        ~~ creditor \in g \mbox{  // check transfer type}
         \+
  \THEN ~ \IF MayStartDebitOpns(to)    \mbox{  // check accont connectivity $to \in CC$}\+
     \THEN ~\IF accountType(from) \in AccT^{Debit,SRD}(accountType(to))
             \mbox{ // i.e. $from \in CC$}\+
       \THEN  ~ \ASM{Completion}(transfer) \\
        \ELSE ~ \ASM{Send}
        (ErrMsg(DebitTargetAccountViolation(transfer)),\TO creditor) \- 
   \ELSE ~ \ASM{Send}
   (ErrMsg(DebitSourceAccountViolation(transfer)),\TO creditor)  \-
  \ELSE ~ \ASM{Send}
  (ErrMsg(DebitTransferTypeViolation(transfer)),\TO creditor) \-
 \ELSE ~ \ASM{Send}
  (ErrMsg(NotAccountOwnerToReceiveDebit(transfer)),\TO creditor)
\end{asm}



\subsection{$\ASM{DebitPerformReq}$ program}
\label{sect:debitperform}

To define $\ASM{DebitPerformReq}$, we reuse the scheme applied for $\ASM{CreditPerformReq}$, calling once more the $\ASM{DebitTypeCheck}$ component executed already by $\ASM{DebitPreviewReq}$, but with a new $\ASM{Completion}$ parameter whose role is to trigger a $\ASM{DebitAccountLimitsCheck}$ and -- if that check succeeds -- a $\ASM{RequestDebitAck}$nowledgement from the $debitor$.   

\begin{asm}
\ASM{DebitPerformReq}(transfer,creditor)  =\+
\IF Received(DebitPerformReq(transfer),\FROM creditor) \THEN \+  
   \ASM{DebitTypeCheck}(transfer,creditor,\ASM{TryToCompleteDebitOpn})\\
   \ASM{Consume}(DebitPerformReq(transfer))\-
\WHERE \+
transfer=(debit,channel,from,to,amount)\\
\ASM{TryToCompleteDebitOpn}(transfer)=\+
\ASM{DebitAccountLimitsCheck}
   (transfer,\ASM{RequestDebitAck}(transfer))
\end{asm}

$\ASM{DebitAccountLimitsCheck}$ is structurally similar to the $\ASM{CreditAccountLimitsCheck}$ (and uses its definitions for the three check predicates), but it has a different parameter to be called in case the check is successful, namely to $\ASM{RequestDebitAck}$nowledgement from the $debitor$ (see below) before completing the transaction either successfully, by a $\ASM{DebitAckReqAnswCompletion}$, or in the failure case by a $\ASM{DebitLateAckReqAnswCompletion}$. 

This leads to the following definition. To prevent confusion we use a new name 
$\ASM{NextStep}$ for the parameter. Note that for privacy reasons, the definition of the content of an $ErrMsg(param)$ (which has to be defined separately) may have to hide some of the information the system knows in case of the given $param$eters.
%\newpage

\begin{asm}
\ASM{DebitAccountLimitsCheck}
               (transfer,\ASM{NextStep})= \+
  \LET (debit,channel,from,to,amount)=transfer \\
  \LET debitor=owner(from). creditor=owner(to)\\
  \IF CanBeSpentBy(cc(debitor),amount) \+
    \THEN ~ \IF CanBeCashedBy(cc(creditor),amount) \+
       \THEN ~ \IF HasSellCapacityFor(amount,cc(creditor)) \+
           \THEN ~ \ASM{NextStep} \\
           \ELSE~\ASM{Send}
           (ErrMsg(SellCapacityViolation(transfer)),
                               \TO creditor\-
       \ELSE~\ASM{Send}
       (ErrMsg(UpperLimitViolation(transfer)),\TO creditor\-
   \ELSE 
    ~ \ASM{Send}(ErrMsg(AvailBalanceViolation(transfer)),\TO creditor
\end{asm}


The machine  $\ASM{RequestDebitAck}$ completes the transaction without further 
ado in case  the $amount$ is $Small$ (less than 100), namely 
by appending it to the $Ledger$ (using the system location 
for the current system time, denoted $now$). For every other 
$amount$, $\ASM{RequestAck}$ creates a $OneTimePassword$ 
$otp$ (using the current system time, denoted by $now$), 
records its birthtime (the beginning of its lifetime), 
records the $otp$ with the transaction (including the 
computed transfer type) as a $PendingTransaction$ and sends 
the $otp$  with an agreement request to the $debitor$. To 
execute $\ASM{DebitAckReqAnswCompletion}$ a $DebitAckMsg$ must 
arrive; if such a message does not arrive within the 
lifetime of $otp$, $\ASM{DebitLateAckReqAnswCompletion}$ will be 
executed.

\begin{asm}
\ASM{RequestDebitAck}(transfer)=\+
\LET (debit,channel,from,to,amount)=transfer \\
\LET debitor=owner(from), creditor=owner(to)\\
\IF Small(amount) 
\mbox{  // case where no acknowledgement from $debitor$ is requested}\+
   \THEN ~ \ASM{CompleteTransaction}(transfer) \-
   \ELSE \+
       \LET otp= \NEW(OneTimePassword)\\
       \LET pendingTransact=(otp,transfer)\+
          birthTime(otp):=now \mbox{ // current system time}\\
          \ASM{Insert}(pendingTransact,PendingTransaction)\\ 
          status(pendingTransact):=pending \\
          \ASM{Send}(ConfirmationReq(pendingTransact),\TO debitor)\footnote{There is no need to keep the $channel$ parameter because the confirmation request can be sent through any channel, not necessarily the one through which the Debit request arrived, and also the acknowledgement can arrive via any channel.}
       \dec\-
\WHERE \+
Small(amount) \IFF amount < 1000 \\
 \ASM{CompleteTransaction}(transfer)\footnote{This machine is structurally the same but differs from the one with the same name used in $\ASM{CreditPerformReq}$ by the debit parameter (instead of credit).} =\+     
   \ASM{Append}(transaction(transfer,ttResult,now),Ledger)\\
   \ASM{Send}(Confirmed(transfer),\TO debitor)\\
   \ASM{Send}(Confirmed(transfer),\TO creditor)\\
   saleVolume(creditor):=saleVolume(creditor)+amount \-
ttResult=(group(creditor),CC,CC)
\end{asm}


\subsection{ $\ASM{DebitAck/RejectCompletion}$ programs}
\label{sect:debitackreject}

\subsubsection{Debit completion when debitor answers $ConfirmationReq$}\label{sect:debitack}

Case 1. When, for a pending transaction $t$, the $ConfirmationReq(t)$ is answered by the $debitor$ but too late, an error message informing that the $otp$ expired is sent to the debitor and the creditor and the message is discarded. In this case, the $\ASM{DebitLateAckReqAnswCompletion}$ rule below will delete the $otp$ and update $status(t)$ to $rejected$.

Case 2. The $ConfirmationReq(t)$ is answered by the $debitor$ in time, i.e. within the $lifetimeForOTPs$ foreseen for one-time passwords, but negatively by a $DebitRejectMsg$. \textcolor{red}{I have inserted this case. I cannot believe that in this case nothing happens, as had been stated explicitly as requirement for the old rule DebitAckReqAnswCompletion (footnote 24 of the version 17.6.2018). That must have been a misunderstanding of ack msg, as if ack means to accept.} In this case the $otp$ is deleted and the pending transaction made $rejected$.

Case 3. The $ConfirmationReq(t)$ is answered by the $debitor$ in time and positively by a $DebitAckMsg$. Then the system will $\ASM{CompleteTransaction}$ and update the transaction status from $pending$ to $performed$, but only after a new $\ASM{FinalDebitAccountLimitsCheck}$ succeeded. Refining the machine $\ASM{DebitAccountLimitsCheck}$ in this way guarantees that, in case of failure, the Debit operation is rejected. Therefore in every case the $status$ of the pending transaction is changed to either $performed$ or $rejected$ so that the one-time password can be deleted,  preventing a later application of the $\ASM{DebitLateAckReqAnswCompletion}$ rule (which will be applied in case of an $Expired(otp)$).


\begin{asm}  
\ASM{DebitAckReqAnswCompletion} =\+           
\IF Received(AnswerMsg(pendingTransact),\FROM debitor)\+
\AND AnswerMsg(pendingTransact) \in \+
 \{DebitAckMsg(pendingTransact),DebitRejectMsg(pendingTransact)\} \THEN \dec \-
\LET (otp, (debit,channel,from,to,amount))=pendingTransact \footnote{Using the $\LET$ clause here relies on the assumption (which can be checked to be true for the model) that $DebitAckMsg$es are formed with the expected correct parameters.}\\
 \LET debitor=owner(from), creditor=owner(to)\+
  \IF Expired(otp) \THEN \+
    \ASM{Send}(ErrMsg(ExpiredOtpFor(DebitAck,amount,creditor)),\TO debitor) \\
    \ASM{Send}(ErrMsg(ExpiredOtpFor(DebitAck,amount,creditor)),\TO creditor)\-
   \ELSE ~\IF   
      Received(DebitRejectMsg(pendingTransact),\FROM debitor) \THEN \+
          \ASM{Delete}(otp,OneTimePassword) \\
          status(t):=rejected\\
          \ASM{Send}(RejectMsg(debit,amount,debitor),\TO creditor)\-         
  \ELSE ~\IF pendingTransact  \in PendingTransaction \AND
    status(pendingTransact) =pending\footnote{Otherwise, following the requirements, the message is just discarded, nothing else happens to the $otp$ or the pending transaction. In D2.1 the possibility to send out an error message was considered.}  \+
       \THEN \+
          \ASM{FinalDebitAccountLimitsCheck}\+
              ~~~~~~ (pendingTransact,\ASM{CompleteTransaction}(pendingTransact))\-
                \ASM{Delete}(otp,OneTimePassword)\dec\dec\-  
\ASM{Consume}(DebitAckMsg(pendingTransact)) \dec\dec\-
\WHERE \+
Expired(otp) \IFF now-birthtime(otp) > lifetimeForOTPs\\
\ASM{CompleteTransaction}(pendingTransact)=\+
   \ASM{CompleteTransaction}(debit,channel,from,to,amount)\\
  status(pendingTransact) :=performed
\end{asm}

The $\ASM{FinalDebitAccountLimitsCheck}$ refines the  $\ASM{DebitAccountLimitsCheck}$ by inserting into the failure cases a clause which makes the pending Debit transaction rejected and informs the creditor about the reason for rejection. Using the $\LET$ clause in the definition relies on the assumption (which can be checked to be true for the model) that each time the submachine $\ASM{FinalDebitAccountLimitsCheck}$ is called in the program, it is called with the expected correct parameters.

\begin{asm}
\ASM{FinalDebitAccountLimitsCheck}(pendingTransact,\ASM{NextStep})=\+
 \LET (otp, transfer)=pendingTransact\\ 
 \LET = (debit,channel,from,to,amount)=transfer\\
  \LET debitor=owner(from), creditor=owner(to)\+
  \IF CanBeSpentBy(cc(debitor),amount) \+
    \THEN ~ \IF CanBeCashedBy(cc(creditor),amount) \+
      \THEN ~ \IF HasSellCapacityFor(amount,cc(creditor)) \+
         \THEN ~ \ASM{NextStep} \\
         \ELSE ~ \ASM{RejectTransactionBecauseOf}
               (SellCapacityViolation,pendingTransact) \-
     \ELSE ~ \ASM{RejectTransactionBecauseOf}
     (UpperLimitViolation,pendingTransact) \-
  \ELSE ~  \ASM{RejectTransactionBecauseOf}
    (AvailBalanceViolation,pendingTransact) \dec\-
\WHERE \+
\ASM{RejectTransactionBecauseOf}(reason,pendingTransact)  =\+
   \ASM{Send}(ErrMsg(reason(transfer)),\TO creditor)\\
   status(pendingTransact):=rejected  
\end{asm}


\subsubsection{Debit rejection upon missing debitor's answer to $ConfirmationReq$}\label{sect:debitreject}

In case the $debitor$ does not confirm the Debit request within the $lifetime$ foreseen for OTPs, the Sardex system will reject the $\ASM{DebitPerformReq}$ (by changing the status of the pending transaction to $rejected$) and inform the $creditor$ about it.\footnote{We use Hilbert's $\iota$ operator notation $\iota x(P)$ to describe the unique element $x$ which satisfies a property $P$, obviously to be used only if there is exactly one such element.}

\begin{asm}  
\ASM{DebitLateAckReqAnswCompletion} =\+           
  \IF otp \in OneTimePassword \AND Expired(otp)  \THEN  \+
  \LET t=\iota t' (t'  \in PendingTransaction \AND 
       t'=(otp,transfer)) \mbox{ // NB. $otp$ is unique}\\
   \LET (debit,channel,from,to,amount)=transfer \+
       \ASM{Delete}(otp,OneTimePassword) \\
      \IF status(t)=pending \THEN \+
         status(t):=rejected\\
         \ASM{Send}(RejectMsg(Debit,amount,debitor),\TO creditor)\\  
         \ASM{Send}(OtpExpired(Debit,amount,creditor),\TO debitor)  \dec \-
  \WHERE debitor=owner(from), creditor=owner(to)
\end{asm}

\section{New B2C operations}
\label{sect:userops}

The new account types $Income$, $Prepaid$, $Bisoo$ and $Topup$ serve for three new operations:
\begin{itemize}
	\item An $\ASM{EurB2C}$ operation triggered by a $consumer$ or a $consumer$\_$verified$ and executed by a $Retail$ business member when either a $consumer$ or a $consumer$\_$verified$ purchases some good and pays in Euro. The operation consists in issuing a reward (in SRD) to the costumer and paying a fee (in EUR) to the Sardex company.
		
	\item An $\ASM{SrdB2C}$ operation triggered by a $Consumer\_Verified$ member and executed by a $retail$er. The operation consists in accepting that for a purchase the member pays the $retail$er by rewards the member accumulated in SRD currency. 
	
	\item  A $\ASM{RechargePrepaid}$ operation executed by the $mngr$, triggered by an input received from a $Retail$ member and declared as $FeePrepayment$ to be paid to the Sardex company for future B2C operations the member may perform with its customers.
	\end{itemize}
All these operations concern exchange of money which we model as $\ASM{Send}$ operations with appropriate parameters.


\subsection{Retail B2C operations ($\ASM{EurB2C}$ and $\ASM{SrdB2C}$)}
\label{sect:eurob2c}

In the refined model there are two new operations, of type B2C (Business to 
Consumer), which are in Euro and SRD currency, respectively.

It seems that these rules are considered part of the Sardex system software and not of software which is executed locally on machines of the business member (in $Retail$ or $Full$). Therefore we describe the rules as triggered by receiving corresponding messages.
%\textcolor{red}{Is this assumption correct? Otherwise we must reformulate the rules below. Paolo: YES, it is correct.}

D3.1 considers the operations as instances of the general Credit/Debit operations. However, these B2C operations do not involve the transaction and account type checks every Credit/Debit operation has to perform. Therefore we simplify the formulation of the rules as rules tailored to perform the necessary dedicated checks and updates, but also to avoid the other general types checks, which are unnecessary here. 

In the current Sardex system, EurB2C operations are triggered by a $retail$ or a $full$ member as Debit operations from the consumer's Bisoo account to the Income account of the $retail$ or $full$ member, respectively. To simplify the description of the desired functionality, in this model we have chosen to describe such operations directly, without the artificial detour via empty account type and account connectivity checks. Thus, an $\ASM{EurB2C}$ operation is triggered by a $consumer$ (whether in $Consumer$ or in $Consumer\_Verified$) who buys a product at a $retail \in Retail \cup Full$ and pays for it in Euro. The money is recorded in the $income(retail)$ account, a reward as SRD credit is issued to the $customer$ and the Euro fee is paid. The consumer from which the $EuroAmount$ is $Received$ remains anonymous, read: until it becomes a $Customer\_Verified$ member, namely by a registration action which we do not model here, it remains known to the Sardex system only by the number of the $card$ issued.

\begin{asm}
\ASM{EurB2C}=\+
\IF Received(EurB2CMsg(EuroAmount,\FROM customer), \FROM retail) \AND \+
          ~~~~~~~~
          customer \in  \{unknown\} \cup Consumer \cup Consumer\_Verified \THEN \\
    \IF ThereIsEnoughPrepaidFeeFor(EuroAmount,retail) \THEN \+
       \ASM{RegisterEuroPayment}(EuroAmount,retail)\\
       \ASM{PayB2CEuroFee}(EuroAmount,\FOR retail) \\ 
       \ASM{IssueReward}(retail,EuroAmount,customer)\-
    \ELSE ~ \ASM{IssueWarning}(ThereIsNotEnoughEuroFeePrepayment)    \\
    \ASM{Consume}(EurB2CMsg(EuroAmount,\FROM customer, \FOR retail)) 
              \mbox{ // consume input} \-
\WHERE \+
    ThereIsEnoughPrepaidFeeFor(amount,retail) \IFF \+    
        balance_{retail}(prepaid_{retail}) \geq euroFee_{retail}(amount) \-
    \ASM{RegisterEuroPayment}(amount,retail)=\+
        balance_{retail}(income_{retail}):= balance_{retail}(income_{retail})+amount \-
    \ASM{PayB2CEuroFee}(amount,\FOR retail)= \+      
        \ASM{Send}(B2CEuroFeeMsg(euroFee_{retail}(amount),\FROM retail), \TO mngr) \-
    \ASM{IssueReward}(retail,amount,customer)  =\+
         balance_{retail}(cc_{retail}):=       
         balance_{retail}(cc_{retail})-rewardRate_{retail}(amount) \\
        \IF customer=unknown \THEN \+
            \LET card=~\NEW{Consumer} \+
               \ASM{InitializeBisoo}(card,amount)  \mbox{ // register Euro payment in new $Bisoo$ account}\\
               \ASM{InitializeReward}(card,rewardRate_{retail}(amount))  \mbox{  // register SRD reward on card} \dec\-
         \ELSE  \mbox{  // $customer$ `is' a $card$ or $customer$ is registered}\+
              \ASM{RefreshBisoo}(customer,amount)   \\
              \ASM{RefreshReward}(customer,rewardRate_{retail}(amount))\dec\-
    \ASM{InitializeBisoo}(card,amount) = 
              (balance_{card}(bisoo_{card}):=amount )    \\           	
    \ASM{InitializeReward}(card,amount)=(balance_{card}(cc_{card}):=amount)\\
    \ASM{RefreshBisoo}(cust,amt)=
            (balance_{cust}(bisoo_{cust}):=balance_{cust}(bisoo_{cust})+amt) \\
     \ASM{RefreshReward}(cust,amt)=
        (balance_{cust}(cc_{cust}):=  balance_{cust}(cc_{cust})+amt)       
\end{asm}
NB. By registering, a $consumer$ becomes a member of $Consumer\_Verified$ whereby its `identity' changes from being a $card$ to a $customer$, with name, surname, etc., and correspondingly its $acc$ounts turn out to be known now as $acc_{customer}$ together with the associated account access function.

{\bf Remark on Bisoo and Income.} For simplicity of exposition, we have included 
the  $bisoo_{consumer}$  action (to record, 
for statistical purposes, the Euro amount of the sale) into  the $\ASM{IssueReward}$ rule. In the 
use case description in D3.1 this action is introduced as an action of the  $mngr$ who 
controls the $Bisoo$ accounts. Note the direct update of $income_{retail}$ in $ \ASM{RegisterEuroPayment}$, which in D3.1 is introduced as a Debit operation involving $income_{retail}$ and $bisoo_{customer}$.


A $retail \in Retail \cup Full$ can also execute a $\ASM{SrdB2C}$ operation which can be triggered by a  member of $Consumer\_Verified$. This happens when the $retail$er accepts a purchase the member pays in SRD via its accumulated rewards. It is assumed that when registering (an operation we do not model here), a consumer is turned from an element $card \in Consumer$ into an element of $Consumer\_Verified$, so that formally the account $cc_{card}$ becomes $cc_{consumerVerified}$. 

\begin{asm}
\ASM{SrdB2C}=\+
   \IF Received(SrdB2CMsg(amount,\FROM consumer), \FROM retail) \AND \+
     ~~~~~~~~~~~~~~~~~~~~~~~~~~~~~~~~~~~~consumer \in Consumer\_Verified\\
 \THEN \+
      \ASM{PayByReward}(amount,consumer,retail)\\
      \ASM{Consume}(SrdB2CMsg(amount,\FROM consumer))\dec\-
 \WHERE \+
\ASM{PayByReward}(amt,consumer,retail)  =\+    
      balance_{consumer}(cc_{consumer}):= 
                balance_{consumer}(cc_{consumer)}) -amt\\
      balance_{retail}(cc_{retail}):= balance_{retail}(cc_{retail}) + amt
\end{asm}

$\ASM{PayByReward}$ is described in D3.1 as an instance of the standard Credit (via Service channel) or Debit (via POS channel) operation. 

\subsection{Mngr/SysAdmin fee operations:\\ $\ASM{RechargePrepaid}$, $\ASM{AcceptFee}$, $\ASM{LowPrepaymentAlert}$}
\label{sect:Prepaid}

In the refined model there are three new $SysAdmin$ operations.

For each $retail \in Retail \cup Full$ user, $SysAdmin$ manages the account $prepaid_{retail} \in Account(retail)$ (which is owned by $retail$ but controlled by $SysAdmin$). This account serves a double purpose: a) to keep track of the fee prepayments made by the $retail$er, in Euro currency, b) to keep track of the fee `consumed' each time the $retail$er performs a $\ASM{EurB2C}$ transaction. 

To control the fee prepayments, when a $retail$er pays an $amount$ of Euros as fee prepayment 
into the $mngr$'s bank account, using any of the standard payment systems, that  $amount$ is added to the $retail$er's $prepaid_{retail}$ account. We describe this in the
$\ASM{RechargePrepaid}$ rule directly, avoiding the detour via a Credit operation performed by the $mngr$ from its auxiliary $topup$ account. This is further justified by the fact that the management of these accounts is not by $mngr$ but, rather, by $SysAdmin$.

For statistical purposes, the fee consumption for each B2C operation performed by a retailer is traced by the $mngr$ by letting the $retail$er `pay the fee  into the $topup$ account'. To do this, $SysAdmin$ moves the amount of Euro  which represents the fee from $prepaid_{retail}$ into $topup$. We describe this in the $\ASM{RecordFeePayment}$ rule below directly, avoiding the detour via a Debit operation performed by the $mngr$ to its auxiliary $topup$ account.

Correspondingly, upon the receipt of a fee prepayment by the retailer through normal banking channels, the prepaid amount is detracted from $topup$.

In addition, before $prepaid_{retail}$ reaches zero, a $lowBalanceAlert$ is sent to 
the $retail$er.


\begin{asm}
\ASM{RechargePrepaid}=\+
\IF Received(EuroFeePrepaymentMsg(amount, \FROM retail)) \AND 
retail \in Retail \cup Full \THEN \+
balance(topup):=balance(topup)-amount \mbox{ // subtract  $amount$ from $topup$}\\
balance(prepaid_{retail}):=balance(prepaid_{retail})+amount \+
\mbox{ // add $amount$ to $prepaid_{retail}$}\-
\ASM{Consume}((amount,FeePrepayment), \FROM retail) \mbox{ // consume input}
\end{asm}


When a $retail$er has to pay the fee for a $\ASM{EurB2C}$ operation it performs with a customer, $SysAdmin$ is triggered to $\ASM{RecordFeePayment}$ when it receives the corresponding $B2CEuroFeeMsg$ from the $retail$er. Following the Euro fee handling scheme via the $topup$ account described above, $SysAdmin$ must add the received fee to $topup$ and subtract it from the $retail$er's $Prepaid$ account.

\begin{asm}
\ASM{RecordFeePayment}=\+   
   \IF Received(B2CEuroFeeMsg(amount,\FROM retail)) \AND retail \in Retail \cup Full \THEN \+
   balance(topup):=balance(topup)+fee \mbox{ // add $fee$ to $topup$}\\
   balance(prepaid_{retail}):=balance(prepaid_{retail}) -  fee 
            \mbox{ // subtract $fee$ from $prepaid_{retail}$}\\
\ASM{Consume}(B2CEuroFeeMsg(amount,\FROM retail))
\end{asm}

\begin{asm}
\ASM{LowPrepaymentAlert}=\+
  \FORALL  retail  \in Retail \cup Full \+
     \IF CloseToZero(balance(prepaid_{retail}))  \THEN \+
        \ASM{Send}(PrepaymentAlertMsg(lowPrepaidBalance),\TO retail)
\end{asm}
\section{User Operations}
\label{sect:usrops}
Users can $\ASM{Send}$ requests which appear as input for the INTERLACE network server. Whether $\ASM{Send}(CreditPreviewReq(transfer))$ or $\ASM{Send}(DebitPreviewReq(transfer))$ is invoked is conditioned only by a correct definition of the $transfer$ parameter, which is a definition the user supplies by filling in the corresponding  fields on the screen. The same holds, \emph{mutatis mutandis}, for $\ASM{Send}(AccountHistReq(histParams))$ and $\ASM{Send}(BalanceReq(acc))$. The functionality is clear so that we do not model further this editing process.

For Credit/Debit Perform requests the only relevant additional constraint is that they can be sent only after an ok-message for the corresponding Preview request has been received. We use a function $kind$ to extract from a $transfer$ parameter its $credit$ or $debit$ component, respectively.\footnote{In the following ASMs the keyword `sardex' stands for `INTERLACE network server'.}

\begin{asm}
\IF Received(YouMayTriggerPerformReq(transfer),\FROM sardex) \THEN \+
\IF kind(transfer)=credit \THEN \+
\ASM{Send}(CreditPerformReq(transfer), \TO  sardex) \-
\IF kind(transfer)=debit \THEN \+
\ASM{Send}(DebitPerformReq(transfer), \TO  sardex)  \-
\ASM{Consume}(YouMayTriggerPerformReq(transfer))
\end{asm}

In case of a Debit operation a debitor has to confirm a received debit request by $\ASM{Send}$ing a $DebitAckMsg$; otherwise a $DebitRejectMsg$ is sent to the INTERLACE network server.
\begin{asm}
\IF Received(ConfirmationReq(otp,transfer),\FROM sardex) \THEN \+
\LET (debit,channel,from,to,amount)=transfer \\
\IF Agreed(amount,owner(to),otp)\+
\THEN ~ \ASM{Send}(DebitAckMsg(otp,transfer), \TO sardex) \\
\ELSE ~ \ASM{Send}(DebitRejectMsg(otp,transfer), \TO sardex)\-
\ASM{Consume}(ConfirmationReq(otp,transfer))
\end{asm}


\section{Sub-Appendix 1: Sardex Business Logic in a Nutshell}
\label{sect:appendixModel}

We assume that both the Credit/Debit/B2C and the $mngr$ operations are executed by the Sardex system with the following program $\ASM{SardexModel}$. Obviously these rules could be split and assigned to different agents, e.g. the last two ones to the $SysAdmin$ and the first two to an independent agent (who by those rules reacts to triggers by the users, but the rules themselves are not under user control).

\begin{asm}
\ASM{SardexOps} =\+
  \ASM{CreditTransferReq}\\
  \ASM{DebitTransferReq}\\
  \ASM{EurB2C}\\
  \ASM{SrdB2C}\\
  \ASM{MngrOps} \-
\WHERE \+
  \ASM{CreditTransferReq}=\+
     \ASM{CreditPreviewReq} \\
     \ASM{CreditPerformReq} \-
  \ASM{DebitTransferReq}=\+
     \ASM{DebitPreviewReq} \\
     \ASM{DebitPerformReq} \\
     \ASM{DebitAckReqAnswCompletion}\\
     \ASM{DebitLateAckReqAnswCompletion}\-
  \ASM{MngrOps} =\+    
     	\ASM{RechargePrepaid}\\
     	\ASM{RecordFeePayment}\\
     	\ASM{LowPaymentAlert}
 \end{asm}
 
 \subsection{The Credit operation components}
 
 Both $\ASM{CreditPreviewReq}$ and $ \ASM{CreditPerformReq}$ rules use the $ \ASM{CreditTypeCheck}$ component defined below.
 
 \begin{asm}
 \ASM{CreditPreviewReq} =\+
    \LET transfer=(credit,channel,from,to,amount)\+
        \IF Received(CreditPreviewReq(transfer),\FROM mbr) \THEN \+   
           \ASM{CreditTypeCheck}(transfer,mbr,\ASM{PermitPerformReq})\\
            \ASM{Consume}(CreditPreviewReq(transfer)) \-
      \WHERE \+
         \ASM{PermitPerformReq}(transfer)=\+
        \ASM{Send}
        (YouMayTriggerPerformReq(transfer),\TO owner(from))
 \end{asm}
 
 
 \begin{asm}
 \ASM{CreditPerformReq} =\+
 \LET transfer=(credit,channel,from,to,amount)\\
 \IF Received(CreditPerformReq(transfer),\FROM mbr) \THEN \+  
 \ASM{CreditTypeCheck}(transfer,mbr,\ASM{TryToCompleteCreditOpn})\\
 \ASM{Consume}(CreditPerformReq(transfer))\-
 \WHERE \+
 \ASM{TryToCompleteCreditOpn}(transfer)=\+
 \ASM{CreditAccountLimitsCheck}(transfer,\ASM{CompleteTransaction}(transfer))
 \end{asm}
 
 \subsubsection{Credit check subcomponents for account types and account limits} 
 
 \begin{asm}
 	\ASM{CreditTypeCheck}(transfer,mbr,\ASM{Completion})  =\+
 	\IF mbr =owner(from) \AND MayAllowTransferForCreditOpns(mbr) \+
 	\THEN ~ \IF ~ \FORSOME g \in TT^{Credit,SRD}(group(mbr)) 
 	~~ owner(to) \in g   \mbox{  // check transfer type}\+
 	\THEN  ~ \IF MayStartCreditOpns(from) 
 	\mbox{  // check account connectivity}\+
 	\THEN ~ \IF ~ accountType(to) \in 
 	AccT^{Credit,SRD}(accountType(from))\+
 	\THEN ~ \ASM{Completion}(transfer) \\
 	\ELSE ~ \ASM{Send}(ErrMsg(CreditTargetAccountViolation(transfer)),\TO mbr) \-
 	\ELSE ~ \ASM{Send}(ErrMsg(CreditSourceAccountViolation(transfer)),\TO mbr)\-
 	\ELSE ~ \ASM{Send}(ErrMsg(CreditTargetGroupViolation(transfer)),\TO mbr) \- 
 	\ELSE ~  \ASM{Send}(NotAccountOwnerOrCreditSourceGroupViolation(transfer),\TO mbr)\dec\dec\-
 	\WHERE \+
 	transfer = (credit, channel, from, to, amount)   
 \end{asm}
 
 \begin{asm}
 \ASM{CreditAccountLimitsCheck}(transfer,\ASM{StepCompletion})=\+
 \IF CanBeSpentBy(from,amount) \+
 \THEN ~ \IF CanBeCashedBy(to,amount) \+
 \THEN ~ \IF HasSellCapacityFor(amount,to) \+
 \THEN ~ \ASM{StepCompletion} \\
 \ELSE 
 ~ \ASM{Send}(ErrMsg(CapacityViolation(transfer)),\TO owner(from))\-
 \ELSE 
 ~ \ASM{Send}(ErrMsg(UpperLimitViolation(transfer)),\TO owner(from))\-
 \ELSE 
 ~ \ASM{Send}(ErrMsg(AvailBalanceViolation(transfer)),\TO owner(from))\dec\-
 \WHERE \+
 transfer=(credit,channel,from,to,amount)\\
 CanBeSpentBy(from,amount) \IFF availableBalance(from) \geq amount \\
 CanBeCashedBy(to,amount) \IFF  
 to \not \in Domu \AND balance(to)+amount \leq upperLimit(to) \\
 HasSellCapacityFor(amount,to) \IFF  to \not \in Domu \AND   amount \leq availableCapacity(to)
 \end{asm}

 \subsubsection{Credit transaction completion subcomponent}
 \begin{asm}
 	\ASM{CompleteTransaction}(transfer)=\+  
 	\LET (credit,channel,from,to,amount)=transfer \\ 
 	\ASM{Append}(transaction(transfer,ttResult,now),Ledger)\\
 	\ASM{Send}(Confirmed(transfer),\TO owner(from))\\
 	saleVolume(to):=saleVolume(to)+amount\footnote{We treat the dynamic function $saleVolume$ as belonging to AccountMetaData, apparently in accordance with D3.1.} \-
 	\WHERE \+
 	ttResult= (group(owner(to)),accountType(from),accountType(to) )
 \end{asm}
 
 \subsection{The Debit operation components}
 

\begin{asm}
	\ASM{DebitPreviewReq}  =\+
	\LET transfer=(debit,channel,from,to,amount)\\
	\IF Received(DebitPreviewReq(transfer),\FROM creditor) \THEN \+   
	\ASM{DebitTypeCheck}
	(transfer,creditor,\ASM{PermitPerformReq})\\
	\ASM{Consume}(DebitPreviewReq(transfer)) \-
	\WHERE \+
	\ASM{PermitPerformReq}(transfer)=\+
	\ASM{Send}
	  (YouMayTriggerPerformReq(transfer),\TO  owner(to)) 
\end{asm}
 

\begin{asm}
	\ASM{DebitPerformReq}  =\+
	\LET 	transfer=(debit,channel,from,to,amount)\\
	\IF Received(DebitPerformReq(transfer),\FROM creditor) \THEN \+  
	\ASM{DebitTypeCheck}(transfer,creditor,\ASM{TryToCompleteDebitOpn})\\
	\ASM{Consume}(DebitPerformReq(transfer))\-
	\WHERE \+
	\ASM{TryToCompleteDebitOpn}(transfer)=\+
	\ASM{DebitAccountLimitsCheck}
	(transfer,\ASM{RequestDebitAck}(transfer))
\end{asm}

 \subsubsection{Debit check subcomponents for account types and account limits}
 \begin{asm}
 	\ASM{DebitTypeCheck}(transfer,creditor,\ASM{Completion})  =\+
 	\LET (debit,channel,from,to,amount) = transfer \\
 	\LET debitor=owner(from)\\
 	\IF creditor=owner(to) \AND MayAllowTransferForDebitOpns(debitor) 
 	\+
 	\THEN~ \IF~\FORSOME g \in TT^{Debit,SRD}(group(debitor))
 	~~ creditor \in g \mbox{  // check transfer type}
 	\+
 	\THEN ~ \IF MayStartDebitOpns(to)    \mbox{  // check accont connectivity $to \in CC$}\+
 	\THEN ~\IF accountType(from) \in AccT^{Debit,SRD}(accountType(to))
 	\mbox{ // i.e. $from \in CC$}\+
 	\THEN  ~ \ASM{Completion}(transfer) \\
 	\ELSE ~ \ASM{Send}
 	(ErrMsg(DebitTargetAccountViolation(transfer)),\TO creditor) \- 
 	\ELSE ~ \ASM{Send}
 	(ErrMsg(DebitSourceAccountViolation(transfer)),\TO creditor)  \-
 	\ELSE ~ \ASM{Send}
 	(ErrMsg(DebitTransferTypeViolation(transfer)),\TO creditor) \-
 	\ELSE ~ \ASM{Send}
 	(ErrMsg(NotAccountOwnerToReceiveDebit(transfer)),\TO creditor)
 \end{asm}

 \begin{asm}
 	\ASM{DebitAccountLimitsCheck}
 	(transfer,\ASM{NextStep})= \+
 	\LET (debit,channel,from,to,amount)=transfer \\
 	\LET debitor=owner(from). creditor=owner(to)\\
 	\IF CanBeSpentBy(cc(debitor),amount) \+
 	\THEN ~ \IF CanBeCashedBy(cc(creditor),amount) \+
 	\THEN ~ \IF HasSellCapacityFor(amount,cc(creditor)) \+
 	\THEN ~ \ASM{NextStep} \\
 	\ELSE~\ASM{Send}
 	(ErrMsg(SellCapacityViolation(transfer)),
 	\TO creditor\-
 	\ELSE~\ASM{Send}
 	(ErrMsg(UpperLimitViolation(transfer)),\TO creditor\-
 	\ELSE 
 	~ \ASM{Send}(ErrMsg(AvailBalanceViolation(transfer)),\TO creditor
 \end{asm}
  
 \begin{asm}
 	\ASM{FinalDebitAccountLimitsCheck}(pendingTransact,\ASM{NextStep})=\+
 	\LET (otp, transfer)=pendingTransact\\ 
 	\LET = (debit,channel,from,to,amount)=transfer\\
 	\LET debitor=owner(from), creditor=owner(to)\+
 	\IF CanBeSpentBy(cc(debitor),amount) \+
 	\THEN ~ \IF CanBeCashedBy(cc(creditor),amount) \+
 	\THEN ~ \IF HasSellCapacityFor(amount,cc(creditor)) \+
 	\THEN ~ \ASM{NextStep} \\
 	\ELSE ~ \ASM{RejectTransactionBecauseOf}
 	(SellCapacityViolation,pendingTransact) \-
 	\ELSE ~ \ASM{RejectTransactionBecauseOf}
 	(UpperLimitViolation,pendingTransact) \-
 	\ELSE ~  \ASM{RejectTransactionBecauseOf}
 	(AvailBalanceViolation,pendingTransact) \dec\-
 	\WHERE \+
 	\ASM{RejectTransactionBecauseOf}(reason,pendingTransact)  =\+
 	\ASM{Send}(ErrMsg(reason(transfer)),\TO creditor)\\
 	status(pendingTransact):=rejected  
 \end{asm}
 
 
 \subsubsection{Debit acknowledgement subcomponent}
 
 \begin{asm}
 	\ASM{RequestDebitAck}(transfer)=\+
 	\LET (debit,channel,from,to,amount)=transfer \\
 	\LET debitor=owner(from), creditor=owner(to)\\
 	\IF Small(amount) 
 	\mbox{  // case where no acknowledgement from $debitor$ is requested}\+
 	\THEN ~ \ASM{CompleteTransaction}(transfer) \-
 	\ELSE \+
 	\LET otp= \NEW(OneTimePassword)\\
 	\LET pendingTransact=(otp,transfer)\+
 	birthTime(otp):=now \mbox{ // current system time}\\
 	\ASM{Insert}(pendingTransact,PendingTransaction)\\ 
 	status(pendingTransact):=pending \\
 	\ASM{Send}(ConfirmationReq(pendingTransact),\TO debitor)
 	\dec\-
 	\WHERE \+
 	Small(amount) \IFF amount < 1000 \\
 	\ASM{CompleteTransaction}(transfer) =\+     
 	\ASM{Append}(transaction(transfer,ttResult,now),Ledger)\\
 	\ASM{Send}(Confirmed(transfer),\TO debitor)\\
 	\ASM{Send}(Confirmed(transfer),\TO creditor)\\
 	saleVolume(creditor):=saleVolume(creditor)+amount \-
 	ttResult=(group(creditor),CC,CC)
 \end{asm}
 
 \subsubsection{Debit completion components (accept/reject transaction)}
 
 
 
\begin{asm}  
	\ASM{DebitAckReqAnswCompletion} =\+           
	\IF Received(AnswerMsg(pendingTransact),\FROM debitor)\+
	\AND AnswerMsg(pendingTransact) \in \+
	\{DebitAckMsg(pendingTransact),DebitRejectMsg(pendingTransact)\} \THEN \dec \-
	\LET (otp, (debit,channel,from,to,amount))=pendingTransact \\
	\LET debitor=owner(from), creditor=owner(to)\+
	\IF Expired(otp) \THEN \+
	\ASM{Send}(ErrMsg(ExpiredOtpFor(DebitAck,amount,creditor)),\TO debitor) \\
	\ASM{Send}(ErrMsg(ExpiredOtpFor(DebitAck,amount,creditor)),\TO creditor)\-
	\ELSE ~\IF   
	Received(DebitRejectMsg(pendingTransact),\FROM debitor) \THEN \+
	\ASM{Delete}(otp,OneTimePassword) \\
	status(t):=rejected\\
	\ASM{Send}(RejectMsg(debit,amount,debitor),\TO creditor)\-         
	\ELSE ~\IF pendingTransact  \in PendingTransaction \AND
	status(pendingTransact) =pending  \+
	\THEN \+
	\ASM{FinalDebitAccountLimitsCheck}\+
	~~~~~~ (pendingTransact,\ASM{CompleteTransaction}(pendingTransact))\-
	\ASM{Delete}(otp,OneTimePassword)\dec\dec\-  
	\ASM{Consume}(DebitAckMsg(pendingTransact)) \dec\dec\-
	\WHERE \+
	Expired(otp) \IFF now-birthtime(otp) > lifetimeForOTPs\\
	\ASM{CompleteTransaction}(pendingTransact)=\+
	\ASM{CompleteTransaction}(debit,channel,from,to,amount)\\
	status(pendingTransact) :=performed
\end{asm}

 
 \begin{asm}  
 	\ASM{DebitLateAckReqAnswCompletion} =\+           
 	\IF otp \in OneTimePassword \AND Expired(otp)  \THEN  \+
 	\LET t=\iota t' (t'  \in PendingTransaction \AND 
 	t'=(otp,transfer)) \mbox{ // NB. $otp$ is unique}\\
 	\LET (debit,channel,from,to,amount)=transfer \+
 	\ASM{Delete}(otp,OneTimePassword) \\
 	\IF status(t)=pending \THEN \+
 	status(t):=rejected\\
 	\ASM{Send}(RejectMsg(Debit,amount,debitor),\TO creditor)\\  
 	\ASM{Send}(OtpExpired(Debit,amount,creditor),\TO debitor)  \dec \-
 	\WHERE debitor=owner(from), creditor=owner(to)
 \end{asm}
 
 \subsection{The B2C operations}
 
 \begin{asm}
 \ASM{EurB2C}=\+
 \IF Received(EurB2CMsg(EuroAmount,\FROM customer), \FROM retail) \AND \+
 ~~~~~~~~
 customer \in  \{unknown\} \cup Consumer \cup Consumer\_Verified \THEN \\
 \IF ThereIsEnoughPrepaidFeeFor(EuroAmount,retail) \THEN \+
 \ASM{RegisterEuroPayment}(EuroAmount,retail)\\
 \ASM{PayB2CEuroFee}(EuroAmount,\FOR retail) \\ 
 \ASM{IssueReward}(retail,EuroAmount,customer)\-
 \ELSE ~ \ASM{IssueWarning}(ThereIsNotEnoughEuroFeePrepayment)    \\
 \ASM{Consume}(EurB2CMsg(EuroAmount,\FROM customer, \FOR retail)) 
 \mbox{~~~~~~~~~ // consume input} 
 \end{asm}
 
 
 \begin{asm}
 \ASM{SrdB2C}=\+
 \IF Received(SrdB2CMsg(amount,\FROM consumer), \FROM retail) \AND \+
 ~~~~~~~~~~~~~~~~~~~~~~~~~~~~~~~~~~~~consumer \in Consumer\_Verified\\
 \THEN \+
 \ASM{PayByReward}(amount,consumer,retail)\\
 \ASM{Consume}(SrdB2CMsg(amount,\FROM consumer))\-
 \WHERE \+
 \ASM{PayByReward}(amt,consumer,retail)  =\+    
 balance_{consumer}(cc_{consumer}):= 
 balance_{consumer}(cc_{consumer)}) -amt\\
 balance_{retail}(cc_{retail}):= balance_{retail}(cc_{retail}) + amt
 \end{asm}
 
 \subsubsection{The B2C operation macros}
 \begin{asm}
 ThereIsEnoughPrepaidFeeFor(amount,retail) \IFF \+    
 balance_{retail}(prepaid_{retail}) \geq euroFee_{retail}(amount) \-
 \ASM{RegisterEuroPayment}(amount,retail)=\+
 balance_{retail}(income_{retail}):= balance_{retail}(income_{retail})+amount \-
 \ASM{PayB2CEuroFee}(amount,\FOR retail)= \+      
 \ASM{Send}(B2CEuroFeeMsg(euroFee_{retail}(amount),\FROM retail), \TO mngr) \-
 \ASM{IssueReward}(retail,amount,customer)  =\+
 balance_{retail}(cc_{retail}):=       
 balance_{retail}(cc_{retail})-rewardRate_{retail}(amount) \\
 \IF customer=unknown \THEN \+
 \LET card=~\NEW{Consumer} \+
 \ASM{InitializeBisoo}(card,amount)  \mbox{ // register Euro payment in new $Bisoo$ account}\\
 \ASM{InitializeReward}(card,rewardRate_{retail}(amount))  \mbox{  // register SRD reward on card} \dec\-
 \ELSE  \mbox{  // $customer$ `is' a $card$ or $customer$ is registered}\+
 \ASM{RefreshBisoo}(customer,amount)   \\
 \ASM{RefreshReward}(customer,rewardRate_{retail}(amount))\dec\-
 \ASM{InitializeBisoo}(card,amount) = 
 (balance_{card}(bisoo_{card}):=amount )    \\           	
 \ASM{InitializeReward}(card,amount)=(balance_{card}(cc_{card}):=amount)\\
 \ASM{RefreshBisoo}(cust,amt)=
 (balance_{cust}(bisoo_{cust}):=balance_{cust}(bisoo_{cust})+amt) \\
 \ASM{RefreshReward}(cust,amt)=
 (balance_{cust}(cc_{cust}):=  balance_{cust}(cc_{cust})+amt)       
 \end{asm}
 
 
  \subsection{The manager operation components}
  
  
 \begin{asm}
 \ASM{RechargePrepaid}=\+
 \IF Received(EuroFeePrepaymentMsg(amount, \FROM retail)) \AND 
 retail \in Retail \cup Full \THEN \+
 balance(topup):=balance(topup)-amount \mbox{ ~~~~~~~~~~~~~// subtract  $amount$ from $topup$}\\
 balance(prepaid_{retail}):=balance(prepaid_{retail})+amount \+
 \mbox{~~~~~~~~~~~~~~~~~~~~~~~~~~~~~~~~~~ 
 	// add $amount$ to $prepaid_{retail}$}\-
 \ASM{Consume}((amount,FeePrepayment), \FROM retail) \mbox{~~~~~~~~~~~~~~~~~~ // consume input}
 \end{asm}
 
 
 \begin{asm}
 \ASM{RecordFeePayment}=\+   
 \IF Received(B2CEuroFeeMsg(amount,\FROM retail)) \AND retail \in Retail \cup Full \THEN \+
 balance(topup):=balance(topup)+fee 
 \mbox{~~~~~~~~~~~~~~~~~~~~~~~~~~~~~~~~~~~~~ // add $fee$ to $topup$}\\
 balance(prepaid_{retail}):=balance(prepaid_{retail}) -  fee 
 \mbox{~~~~~~~~~~~ // subtract $fee$ from $prepaid_{retail}$}\\
 \ASM{Consume}(B2CEuroFeeMsg(amount,\FROM retail))
 \end{asm}
 
 \begin{asm}
 \ASM{LowPrepaymentAlert}=\+
 \FORALL  retail  \in Retail \cup Full \+
 \IF CloseToZero(balance(prepaid_{retail}))  \THEN \+
 \ASM{Send}(PrepaymentAlertMsg(lowPrepaidBalance),\TO retail)
 \end{asm}


\section{Sub-Appendix 2: $\ASM{IfThenElseCascade}$ Pattern}
\label{sect:appendix}

The pattern, used for $\ASM{CreditPreviewReq}$ and $\ASM{CreditPerformReq}$, is obtained by an iteration of the following machine $\ASM{IfThenElse}$:

\begin{asm}
	\ASM{IfThenElse}(Cond,M,N)=\+ 
	\IF Cond \+
	\THEN M \\
	\ELSE N
\end{asm}

This machine is applied to conditions $Cond_i$ and machines $\ASM{Yes}_i$ and $\ASM{No}_i$  where $\ASM{Yes}_i$ is again an $\ASM{IfThenElse}$:
\[\ASM{Yes}_i=\ASM{IfThenElse}(Cond_{i+1},\ASM{Yes}_{i+1},\ASM{No}_{i+1})\]

Given a family $IFE=(IFE_i)_{1 \leq i \leq n+1}$ of conditions $Cond_i$ with ASMs $\ASM{Yes}_i$ and $\ASM{No}_i$,
the pattern machine $\ASM{IfThenElseCascade}(IFE)$ can be defined recursively as follows, starting for $n=1$ with $\ASM{IfThenElse}(Cond_1,\ASM{Yes}_1,\ASM{No}_1)$:


\begin{asm}
	\ASM{IfThenElseCascade}((IFE_i)_{1 \leq i \leq n+1})=\+
	\ASM{IfThenElse}(Cond_1,\ASM{IfThenElseCascade}(IFE_i)_{2 \leq i \leq n+1},\ASM{No}_1)     
\end{asm}






