\chapter{Introduction to the CoreASIM Language, Interpreter, and ICEF}
\label{ch:CoreAsimIntro}

\vspace{-1cm}
\begin{center}
Eduard Hirsch and Paolo Dini
\end{center}

In this chapter an overview of the used technologies for running the INTERLACE Specification is given. At the beginning a quick start description is provided in order to jump right into the execution of the model.

Further details which are enabling the Abstract State Interacting Machines (ASIM) specifications to be executed are discussed and why this possible in a simple and stable manner.

\section{Quick Start Vagrant}
\label{sec:quick-start-vagrant}

There are two base environments available. One based one docker and one based on vagrant. The focus shifted from the vagrant environment which can be downloaded at github\footnote{https://github.com/InterlaceProject/ASIMVagrantEnvironment} to a docker based version which is explained in section \ref{sec:quick-start-docker}. Nevertheless, for developer preferring vagrant, the setup will be still explained.

The vagrant definition provides a running environment for executing the INTERLACE ASIM definitions. During the provisioning process an ubuntu vagrant box is set up which installs the necessary components on that box. It is cloning and building the ICEF framework\footnote{https://github.com/biomics/icef} as well as the ASIM Specification\footnote{https://github.com/InterlaceProject/ASIMSpec} into the data directory where it is finally ready for usage.

\subsection{Prerequisites}

Download and install the following software products:

\begin{itemize}
	\item Virtual Box: https://www.virtualbox.org/
	\item git: https://git-scm.com/downloads
	\item vagrant: https://www.vagrantup.com/
\end{itemize}

\subsection{Clone Environment}

To clone the ASIM vagrant environment from github into a directory git can be utilized:

\begin{lstlisting}
	git clone https://github.com/InterlaceProject/ASIMVagrantEnvironment.git
\end{lstlisting}

\subsection{Execution}

Once all software components are installed and the vagrant definitions are cloned it is possible to call

\begin{lstlisting}
	execute.sh
\end{lstlisting}

from the main directory in order to let the INTERLACE specifications run. Note: when using Windows it is necessary to start that command within git-bash which needs to run in elevated admin mode (right click $\rightarrow$ start as administrator).

On the very first execution the script is provisioning a virtual machine based on ubuntu by calling \texttt{vagrant up} which may take some time. Consecutive calls will be much faster. A detailed description explaining the precise process is covered in section \ref{sec:env-exec}.

Once the execution is started it will run until it is stopped by pressing \textbf{ctrl + c} or by calling

\begin{lstlisting}
	stop.sh
\end{lstlisting}

from any other console window.

\section{Quick Start Docker}
\label{sec:quick-start-docker}

The docker project at github\footnote{https://github.com/InterlaceProject/ASIMDockerEnvironment} is also based on virtualization like the vagrant environment but emphasizing \textit{Operating System} instead of \textit{Hardware virtualization}\footnote{https://www.docker.com/what-container\#comparing}.

\subsection{Installation}

\begin{itemize}
	\item install docker
	\item install git (including git bash for windows)
\end{itemize}

On Linux machines it is important to add the current user to the docker group in order to manage docker container and images. Otherwise all further explained commands need to be executed as root or with sudo.

\subsection{Before First Execution}

In order to configure the environment it is necessary to call the following script:

\begin{lstlisting}
	./configure
\end{lstlisting}

This will generate a docker container image called \textit{asim} where all the necessary frameworks are build and prepared for execution of the specifications. The ICEF framework as well as the ASIM model specifcations are cloned outside of the container to simplify development.

\subsection{Execute Specification}

The container image \textit{asim} created during the configuring step can be started by calling

\begin{lstlisting}
	./execute
\end{lstlisting}

A container started in this way is called \textit{active\_asim} and running all the necessary steps like starting an ICEF manager as well a ICEF brapper to run the ASIM specifications.

Like the vagrant environment a running execution may be stopped by pressing \textbf{ctrl + c}.

\section{Execution Environment Stack}
\label{sec:exec-env-stack}

Independent of the used virtualization techniques a consistent base system is used. So both docker as well as vagrant are provisioning a Linux based operating system using an Ubuntu 16.04 LTS (Long Term Support) distribution.
That consistent, stable and reliable structure will be important later when considerations about provability as well testability come into place. That design provides always the same preconditions and everybody executing or testing against the specifications will receive the same results.

\subsection{Software Stack}

The Ubuntu 16.04 LTS distribution is enhanced and updated according to the needs of an ASIM executing machine as well as to the needs of developers working with that virtual system. To be more specific the following components are installed during the provisioning process:

\begin{itemize}
	\item curl $\rightarrow$ Tool for querying REST resources (used for downloading package resources)
	\item nodejs $\rightarrow$ JavaScript engine including the package manager npm (running the Manager Component of ICEF)
	\item build-essential $\rightarrow$ Packages needed to compile a debian based package (used for building the project sources)
	\item maven $\rightarrow$ Java build and packaging tool (used for building the project sources)
	\item vim $\rightarrow$ Well known U/Linux editor (for development purposes)
	\item git $\rightarrow$ Distributed Version Control System (downloading source repositories from GitHub)
	\item Java 8 $\rightarrow$ Programming Language used for coreASIM base system implementation (running ASIM instances)
\end{itemize}



\todo{Ubuntu, Manager, Brapper, Refs, Figures, Development}

\section{ICEF - The Interaction Computing Execution Framework}
\label{sec:icef-intro}

\todo{references}

The interaction framework is wrapping the original coreASM framework in order to extend it and giving it the capabilities to execute concurrent and distributed. It has been developed in a project called BIOMICS and financed by the European Comission.

This wrapping took place on three levels:

First the interpreter coreASM had to be extended supporting additional languages primitives as well as communications features. Here BSL replaces ASM as a new language having a new interpreter coreASIM.

Second a place has been created were the now so called Abstract State Interaction Machines (ASIMs) take over and are able to execute in parallel. This environment is called Brapper (short for BIOMICS Wrapper).

Third a central server called manager takes care handling distributed Brapper instances dealing with message and scheduling issues.

Technology wise coreASIM changes the coreASM implementation in Java. Brappers are written in Java as well but the managers are coordinating the Brappers using nodejs.

\section{coreASIM}

\section{Interpreter}

\section{INTERLACE Model Execution Environment}
\label{sec:env-exec}

\section{keep for later usage}
\todo{section remove later}
What will happen:
\begin{itemize}
	\item On the very first execution the script is provisioning a virtual machine based on ubuntu by using vagrant up
	\item If the virtual machine is not yet running it tries to start the virtual machine.
	\item If the VM is finally running the script data/execute.sh is called on the guest vm.
\end{itemize}

The actual execution is running inside of the Virtual Machine:

The script data/execute.sh starts the icef manager as well as one brapper. Then it is submitting the specification (run.icef) located in data/ASIMSpec/ to the manager. When done the script is waiting for a stop command or may also be stopped using \textbf{ctrl + c}.

\todo{sections structure, text}