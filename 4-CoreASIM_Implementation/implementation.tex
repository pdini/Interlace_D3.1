\chapter{CoreASIM Implementation of the INTERLACE Business Logic}
\label{ch:CoreAsimImplementation}

\vspace{-1cm}
\begin{center}
Eduard Hirsch
\end{center}

This chapter is taking a look at the ASIM implementation of the Interlace Project which have been done according to the specifications of D2.1 as well as the refinement specifications of the requirements.

In the previous chapter \ref{ch:CoreAsimIntro} the basis for an environment and code execution has been discussed. Further it will be described here how that environment has been utilized. The following detailed implementation design will show what issues and difficulties had to been taken into account.


\section{Introduction}
\label{sec:impl_intro}

The requirements specifications are the basis for the implementation of an executable ASIM model. That model can be found on GitHub \footnote{https://github.com/InterlaceProject/ASIMSpec} will act as a foundation and test/verification template for further business implementations.

This main part of that document is now focusing on the way the ASIM model actually was implemented and how the missing functional parts of the back-end are realized which is mainly about simulating a simple ledger.

\section{Agents}

The implementation base is founded on several ASIM-agents which are programmed to act independent and communicate with each other by the base messaging system provided by the ICEF infrastructure.

There are small differences between the available ASIM-agents and they can be, base on their purpose, categorized into the following three groups:

\begin{itemize}
	\item full agents
	\item dynamic agents
	\item non-functional agents
\end{itemize}

\textit{Full agents} are started right away from the beginning. They are processing request or are handling other duties over the environment lifetime. \textit{Dynamic agents} are created during a specific phase of a test and destroyed after that test has been completed. \textit{Non-functional agents} are never started directly and are facilitated for integration into another agent. They are hosting initialization code or helper functions in order to compensate for the missing modularization feature of the ASIM-BSL language.

To get more familiar with the actual ASIM realization an extract of available agents is listed in table \ref{tab:model-agents}. Those may act as a template for extending the scenario applying additional use cases or tests. They are covering the core payment functionality, thus credit as well as debit operations.

\begin{table}[H]
\begin{centering}
\small
{
\begin{tabular}{ r | p{9cm} | l }
\hline
\textbf{Agent}	& Function & Type \\
\Xhline{1.5pt}
$scheduler$				& Scheduling, creating and destroying dynamic agents & full\\[3pt]
\hline
$server$				& Server which talks to the dynamic clients in order to handle payment and ledger specific requests & full\\[3pt]
\hline
$CreditRequestClient$	& Handles a test credit request & dynamic\\[3pt]
\hline
$DebitRequestClient$	& Initiating a test debit request & dynamic\\[3pt]
\hline
$DebitAcknoledgeClient$	& Confirming a test debit request & dynamic\\[3pt]
\hline
$initdata$				& Fake database and back-end initalization code to be included into the \textit{server} agent & non-functional\\[3pt]
\Xhline{1.5pt}
\end{tabular}
}
\caption{\small\textbf{Agent list and their respective functionality}}
\label{tab:model-agents}
\end{centering}
\vspace{-0.5cm}
\end{table}

\section{Execution}

As mentioned in chapter \ref{ch:CoreAsimIntro} the specifications are executed using the ICEF framework. There is has been covered how a JSON ICEF file can be loaded and started. Now details are given how that process is working in depth particular for Interlace specification.

The whole process starts with the main ICEF definition file in the ASIMSpec directory called run.icef by definition and is shown in Listing \ref{lst:interlace-json-spec}. It is illustrating how the different types of agents are included into the simulation. To explain further all agents are located in the directories \textit{casim} together with \textit{casim/clients}. Agent file names are the same as described in table \ref{tab:model-agents} including the suffix ".casim". \textit{casim} contains all full plus their non-functional agents which are joined later on. \textit{casim/clients} hosts all the relevant clients talking to a server. Currently there is one client for a credit request and two clients for a debit request.

\begin{center}
\begin{minipage}{0.8\textwidth}
\small
\begin{lstlisting}[language=json,firstnumber=1,caption={\bf\small ICEF JSON Specification for INTERLACE},captionpos=b,label=lst:interlace-json-spec]
{
    "id": "interlace", 
    "schedulers": [{
        "file": "casim/scheduler.casim",
        "include": [
            "casim/clients/CreditRequestClient.casim",
            "casim/clients/DebitRequestClient.casim",
            "casim/clients/DebitAcknowledgeClient.casim"
        ],
        "start": "true"
    }],
    "asims": [{
        "file": "casim/server.casim",
        "include": [
            "casim/initdata.casim"
        ],
        "start": "true"
    }]
}
\end{lstlisting}
\end{minipage}
\end{center}

When looking closer at the icef definition one can see that there are couple of agents defined inside of a json-array with an attribute called \textit{include}. All agents in that array will be added to the main file. That means that CreditRequestClient, DebitRequestClient and DebitAcknowledgeClient are added to the scheduler and initdata is added to server agent.

\textbf{Important note:} The include syntax is used to append code from a different file to an agent but comes with strings attached. See \ref{sec:impl-include} for further details.

Consequently the loading module will assemble no more than two ASIMs - a scheduler and a server. That assembled JSON String is sent to the manger which is initializing the simulation environment and distributing the clients over the available brappers.

\subsection{Main Agents Tasks}
\label{subsec:impl-agent-tasks}

As mentioned in the beginning of that section two main agents, scheduler and server, are the core of the running environment. Once started both instantly start working. The server listens for messages on the communication channel in order to process potential request and the scheduler is initializing the first test and starting clients which should be active during that particular test. When a client has been initialized it will carry out its assigned duties which will be to assemble a request to the server and handle the resulting query-response traffic. Of course the request types are of different types depending on the current test and the type of the client. When the client is done it is sending a message to the scheduler which will terminate the client.

\subsubsection{The scheduler}\ as quickly introduced before is responsible for spawning new clients which are taking over different tasks. To do so the scheduler is defining various things in advance. First the current tests need to be identified which has been done by creating a universe as well as two location:

\begin{lstlisting}[language=bsl]
	universe TEST_STATE = { TEST_CREDIT, TEST_DEBIT }
	controlled currentTest: TEST_STATE
	controlled nextTest: TEST_STATE -> TEST_STATE
\end{lstlisting}

The universe \textit{TEST\_STATE} defines the available test list. \textit{currentTest} is set from the the universe \textit{TEST\_STATE} and is defining the currently executed test. Finally \textit{nextTest} is a function which takes as parameter the current test as type \textit{TEST\_STATE} and gives you as result the next test of the queue which is of the same type. To initialized the predefined locations the following command are executed:

\begin{lstlisting}[language=bsl]
	currentTest := START
	nextTest(START) := TEST_CREDIT
	nextTest(TEST_CREDIT) := TEST_DEBIT
\end{lstlisting}

Now it is possible during the execution of the schedulers main program to transition from one test to the next by

\begin{lstlisting}[language=bsl]
	currentTest := nextTest(currentTest)
\end{lstlisting}

applying the current test to the nextTest function. That can be done as long as the return value of the function is \textit{undef} implying that no other test is left.

For the \textit{current test} a functional location is applied which returns a list of client agent for a given test state which can be defined as follows:

\begin{lstlisting}[language=bsl]
	controlled nextClient: TEST_STATE -> LIST
\end{lstlisting}

That \textit{nextClient} function needs to be setted up during the initialization phase of the agent as well

\begin{lstlisting}[language=bsl]
	nextClient(TEST_CREDIT) := [ CreditRequestClient ]
	nextClient(TEST_DEBIT)  := [ DebitAcknowledgeClient, DebitRequestClient ]
\end{lstlisting}

showing that we have one client for a credit request test and two clients for a debit request. Also important to remember is that rules for a client agent are initially defined in a separate file but as noted already included into the scheduler later by just appending the code. Thus each tested client is absolutely required to have unique names for the \textit{initialization}, the \textit{program} and the \textit{policy} rule \textbf{over all} included files!

Next the missing parts are added up to get the full picture on how the clients are started. In particular in which way the base rules for a dynamic client are defined and how they are finally instantiated and handled during runtime. Listing \ref{lst:impl-sched-core} is illustrating the management of the different clients.

\begin{center}
\begin{minipage}{0.8\textwidth}
\small
\begin{lstlisting}[language=bsl_lst,caption={\bf\small Scheduler core functionality},label={lst:impl-sched-core} ]
	//definition of functional location for getting
	//a function reference for a client name
	controlled initBy: CLIENT -> FUNCTION
	controlled withProg: CLIENT -> FUNCTION
	controlled andPol: CLIENT -> FUNCTION
	
	rule Start = {
		...
		//Setup CreditRequestClient rules
		initBy(CreditRequestClient) := InitCreditRequestClient
		withProg(CreditRequestClient) := ProgramCreditRequestClient
		andPol(CreditRequestClient) := SkipCreditRequestClient
		...
	}
	
	rule Program = {
		...
		//Client rules are defined in clientTemplate script
		if currentTest != undef then seq			
			forall createClient in nextClient(currentTest) do
				createASIM createClient
					initializedBy initBy(createClient)
					withProgram withProg(createClient)
					andPolicy andPol(createClient)
					in activeList(createClient)
					
			activeClients := | nextClient(currentTest) |
		endseq
		...
	}
\end{lstlisting}
\end{minipage}
\end{center}

In that listing we can identify three parts. \textbf{First} the definition of three functions which are taking a $CLIENT$ as parameter and returning a $FUNCTION$ can be observed, namely $initBy$, $withProg$ and $andPol$. That $FUNCTION$ return type can be any rule we have defined inside of the scheduler or inside of the dynamic clients which are included to the scheduler. \textbf{Second} for client $CreditRquestClient$ those three function values are set. So for example it is possible to apply $initBy(CreditRequestClient) := InitCreditRequestClient$ and afterwards to call function $InitCreditRequestClient$ by using  $initBy(CreditRequestClient)()$. \textbf{At last} during the iterative execution of the $Program$ rule the actual starting of the client is process. If a valid test has been selected $nextClient(currentTest)$ provides a list of clients valid during that particular test. The $forall$ loop is walking through that list and using the respective iteration to instantiate the chosen client. This instantiation is done by calling $createASIM$ command and passing the start up rule ($initializedBy$), the main program rule ($withProgram$) and the scheduling policy ($andPolicy$).

In line 27 of listing \ref{lst:impl-sched-core} the count of active clients is stored. That count is reduced by one when a "Done" message has been received by a dynamic client. Also the client is shut down by calling

\begin{lstlisting}[language=bsl]
	destroyASIM clientName
\end{lstlisting}

After the $activeClients$ count has been reduced to $0$ again all clients have terminated and the next test can be covered. To conclude it is possible to say that $activeClients$ count is used as a structure similar to semaphores which is taking care that no new clients are issued as long as it has a value bigger than $0$ and the scheduler is therefore in something similar to a paused state. Certainly, is it not really paused but in a "busy wait loop".

\subsubsection{The server}\ agent acts a the main component which is implementing most of the specified requirements, which is for the moment credit and debit operations. The server is started right away when CASIMA has received and processed the JSON file for the simulation.

When the init-rule called $Start$ is executed the $Ledger$ and $PendingTransaction$ are initialized as empty maps, an OTP lifetime is set as well as the $Logger$ is setted up. The logger offers several logging levels. Details can be found in section \ref{sec:impl-log}. Also a rule called InitData() is executed at start-up. It is important to know that the Eclipse Plug-in is showing that rule as Problem/Error because it is not defined in the server.casim file and thus not recognizable for it. Nevertheless, that rule is placed correctly and will work fine, because it is defined in the non-functional agent $initdata$. That $initdata$ content is, due to the include inside of the run.icef, appended to the server.casim before it is sent to manger.

As a consequence the rest of the initialization is consolidated inside that rule. To be more specific the following locations are prepared for later usage:

\begin{itemize}
	\item $sessionData$ ... simulates session information
	\item $profileTable$ ... contains user profiles
	\item $accountTable$ ... user accounts
	\item $userGroupTable$ ... specifying users group membership
	\item $TT$ ... transfer type translation table
	\item $accountConnectivity$ ... transferability between accounts
\end{itemize}

\section{Modularization and Include Syntax}
\label{sec:impl-include}

For the Interlace specification the ICEF framework has been extended to support an "include" syntax inside of the ICEF JSON files in order to import sources to an agent. This has been added in order to compensate for the "Modularity" module of ASM which stops working in ASIM because of its distributed nature.

However, it is important to know that the include statement needs to be used with caution, because one need to be aware that it is nothing more but appending the content of the included file to the main agent file. This appending raises the following issues:

\begin{itemize}
	\item Line numbers are different to the original files
	\item For a compilation problem it might be necessary to take a look at all files, main as well as included ones
	\item Naming needs to be consistent throughout all files. E.g. Eclipse will not notify a developer whether a name for a rule, location, ... has been used twice.
\end{itemize}

Nevertheless, it is an approach for handling the code separation in order to avoid a single and extremely long file which would be difficult to maintain and work with.

For being able to use \textbf{Eclipse} and the ASIM eclipse hinting/error detection provided by the plugin another quick fix has been introduced. So if you'd add a dynamic or non-function agent it will be appended at some point to a parent agent. Thus definitions like "CoreASIM asimname" would occur twice inside of that final agent. For the interpreter to work correctly it is consequently necessary to have only one header defining the name of an agent and to remove that header definition inside of the included files. But when you are removing the header definition Eclipse is not able any more to provide correct syntax highlighting, hinting or error detection.

Thus you now have the possibility to mark a section which will be removed during the agent assembling. The beginning of such a section is marked with \textcolor{eclipseComment}{/*includeskip begin*/} and the end with \textcolor{eclipseComment}{/*includeskip end*/}. When using those markers they need to be placed exactly as described - no additional white-spaces (space, tab, return, ...) or different casing. Listing \ref{lst:includeskip} is showing a short example.

Inside of the skipped section there may be many different things placed as shown in the example in order to work seamlessly with the Eclipse hinting. So you could also place names of locations or universes. The reason putting them there is to avoid a warning by the Eclipse plugin that the variable/location has not been defined yet. So in order to check correct spelling and avoid the problems cause by those only getting obvious at interpretation time.

\begin{center}
\begin{minipage}{0.8\textwidth}
%\small
\begin{lstlisting}[language=bsl_lst,caption={\bf\small includeskip usage},label=lst:includeskip]
/*includeskip begin*/
//this part will be removed
CoreASIM Company

use Standard 
init dummy
rule dummy = skip
scheduling NoPolicy
policy NoPolicy = skip
/*includeskip end*/

//this rule is included to main agent
rule somerule = {
	...
}
\end{lstlisting}
\end{minipage}
\end{center}

\section{Communication and Message passing}
\label{sec:impl-com-msg}

message structure send
message passing

\section{Logging}
\label{sec:impl-log}

server logger

\section{Dynamic Clients}
\label{sec:impl-dyn-clients}

dynamic clients
how are agents created, used, destroyed ... example

\section{State Management}
\label{sec:impl-states}

state management (client \& server)

\section{Simple Test Scenario}
\label{sec:impl-test}

explain test cases and scenarios

\section{Database and Agent Initialization}
\label{sec:impl-db-init}

init data
temp table structure
backend simulation

\section{Additional Login Layer}
\label{sec:impl-login-layer}

active login how handled

\section{Important Rule}
\label{sec:impl-rules}

explain helper rules not defined in requirements nor in other sections

\section{Issues}
\label{sec:impl-issues}

function conversion/translation, bugs??


 
  
  

