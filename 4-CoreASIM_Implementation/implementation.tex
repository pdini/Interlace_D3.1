\chapter{CoreASIM Implementation of the INTERLACE Business Logic}
\label{ch:CoreAsimImplementation}

\vspace{-1cm}
\begin{center}
Eduard Hirsch
\end{center}

This chapter is taking a look at the ASIM implementation of the Interlace Project which have been done according to the specifications of D2.1 as well as the refinement specifications of the requirements.

In the previous chapter \ref{ch:CoreAsimIntro} the basis for an environment and code execution has been discussed. Further it will be described here how that environment has been utilized. The following detailed implementation design will show what issues and difficulties had to been taken into account.


\section{Introduction}
\label{sec:impl_intro}

The requirements specifications are the basis for the implementation of an executable ASIM model. That model can be found on GitHub \footnote{https://github.com/InterlaceProject/ASIMSpec} will act as a foundation and test/verification template for further business implementations.

This main part of that document is now focusing on the way the ASIM model actually was implemented and how the missing functional parts of the back-end are realized which is mainly about simulating a simple ledger.

\section{Agents}

The implementation base is founded on several ASIM-agents which are programmed to act independent and communicate with each other by the base messaging system provided by the ICEF infrastructure.

There are small differences between the ASIM-agents and they can be therefore categorized into the following three groups:

\begin{itemize}
	\item full agents
	\item dynamic agents
	\item non-functional agents
\end{itemize}

\textit{Full agents} are started right away from the beginning. They are processing request or are handling other duties over the environment lifetime. \textit{Dynamic agents} are created during a specific phase of a test and destroyed after that test has been completed. \textit{Non-functional agents} are never started directly and are facilitated for integration into another agent. They are hosting initialization code or helper functions in order to compensate for the missing modularization feature of the ASIM-BSL language.

To get more familiar with the actual ASIM realization an extract of available agents is listed in table \ref{tab:model-agents}. Those may act as a template for extending the scenario for additional use cases or tests. They are covering the core payment functionality, thus a credit as well as a debit operation.

\begin{table}[H]
\begin{centering}
\small
{
\begin{tabular}{ r | p{9cm} | l }
\hline
\textbf{Agent}	& Function & Type \\
\Xhline{1.5pt}
$scheduler$				& Scheduling, creating and destroying dynamic agenst & full\\[3pt]
\hline
$server$				& Server which talks to the dynamic clients in order to handle payment and ledger specific requests & full\\[3pt]
\hline
$CreditRequestClient$	& Handles a test credit request & dynamic\\[3pt]
\hline
$DebitRequestClient$	& Initiating a test debit request & dynamic\\[3pt]
\hline
$DebitAcknoledgeClient$	& Confirming a test debit request & dynamic\\[3pt]
\hline
$initdata$				& Fake database and back-end initalization code to be included into the Server agent & non-functional\\[3pt]
\Xhline{1.5pt}
\end{tabular}
}
\caption{\small\textbf{Agent list and the respective functionality}}
\label{tab:model-agents}
\end{centering}
\vspace{-0.5cm}
\end{table}


\section{temp}

brainstorm
  icef description
    server, scheduler (test list), initdata
    Clients: CreditRequestClient, DebitAcknoledgeClient, DebitRequestClient
  tests  
  message structure send
  message passing
  logging
  dynamic clients
  state management (client \& server)
  init data
  include load enhancements, incl. duplicate variable definition problem  
  function conversion/translation
  temp table structure
  backend simulation
  
missing modularization feature